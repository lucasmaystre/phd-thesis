%%%%%%%%%%%%%%%%%%%%%%%%%%%%%%%%
\section{Conclusion}
\label{sec:conclusion}

In this paper, we develop a stationary-distribution perspective on the maximum-likelihood estimate of Luce's choice model.
This perspective explains and unifies several recent spectral algorithms from an ML inference point of view.
We present our own spectral algorithm that works on a wider range of data, and show that the resulting estimate significantly outperforms previous approaches in terms of accuracy.
We also show that this simple algorithm, with a straighforward adaptation, can produce a sequence of estimates that converge to the ML estimate.
On real-world datasets, our ML algorithm is always faster than the state of the art, at times by up to two orders of magnitude.

Beyond statistical and computational performance, we believe that a key strength of our algorithms is that they are simple to implement.
As an example, our implementation of \LSR{} fits in ten lines of Python code.
The most complex operation---finding a stationary distribution---can be readily offloaded to commonly available and highly optimized linear-algebra primitives.
As such, we believe that our work is very useful for practitioners.
