%%%%%%%%%%%%%%%%%%%%%%%%%%%%%%%%
\section{Introduction}
\label{fi:sec:intro}

Aggregating pairwise comparisons and partial rankings are important problems with applications in econometrics \citep{mcfadden1973conditional}, psychometrics \citep{thurstone1927method, bradley1952rank}, sports ranking \citep{plackett1975analysis, elo1978rating} and multiclass classification \citep{hastie1998classification}.
One possible approach to tackle these problems is to postulate a statistical model of discrete choice.
In this spirit, \citet{luce1959individual} stated the \emph{choice axiom} in a foundational work published over fifty years ago.
Denote $p(i \mid A)$ the probability of choosing item $i$ when faced with alternatives in the set $A$.
Given two items $i$ and $j$, and any two sets of alternatives $A$ and $B$ containing $i$ and $j$, the axiom posits that
\begin{align*}
\frac{p(i \mid A)}{p(j \mid A)}
= \frac{p(i \mid B)}{p(j \mid B)}.
%p(i \mid A) / p(j \mid A)
%= p(i \mid B) / p(j \mid B).
\end{align*}
In other words, the odds of choosing item $i$ over item $j$ are independent of the rest of the alternatives.
This simple assumption directly leads to a unique parametric choice model, known as the Bradley--Terry model in the case of pairwise comparisons, and the Plackett--Luce model in the generalized case of $k$-way rankings.
In this paper, we highlight a connection between the maximum-likelihood (ML) estimate under these models and the stationary distribution of a Markov chain parametrized by the observed choices.
Markov chains were already used in recent work \citep{dwork2001rank, negahban2012iterative, azari2013generalized} to aggregate pairwise comparisons and rankings.
These approaches reduce the problem to that of finding a stationary distribution.
By formalizing the link between the likelihood of observations under the choice model and a certain Markov chain, we unify these algorithms and explicate them from an ML inference perspective.
We will also take a detour, and use this link in the reverse direction to give an alternative proof to a recent result on the error rate of the ML estimate \cite{hajek2014minimax}, by using spectral analysis techniques.

Beyond this, we make two contributions to statistical inference for this model.
First, we develop a simple, consistent and computationally efficient spectral algorithm that is applicable to a wide range of models derived from the choice axiom.
The exact formulation of the Markov chain used in the algorithm is distinct from related work \citep{negahban2012iterative, azari2013generalized} and achieves a significantly better statistical efficiency at no additional computational cost.
Second, we observe that with a small adjustment, the algorithm can be used iteratively, and it then converges to the ML estimate.
An evaluation on five real-world datasets reveals that it runs consistently faster than competing approaches and has a much more predictable performance that does not depend on the structure of the data.
The key step, finding a stationary distribution, can be offloaded to commonly available linear-algebra primitives, which makes our algorithms  scale well.
Our algorithms are intuitively pleasing, simple to understand and implement, and they outperform the state of the art, hence we believe that they will be highly useful to practitioners.

The rest of the paper is organized as follows.
We begin by introducing some notations and presenting a few useful facts about the choice model and about Markov chains.
By necessity, our exposition is succinct, and the reader is encouraged to consult \citet{luce1959individual} and \citet{levin2008markov} for a more thorough exposition.
In Section~\ref{fi:sec:relwork}, we discuss related work.
In Section~\ref{fi:sec:algorithms}, we present our algorithms, and in Section~\ref{fi:sec:experimental} we evaluate them on synthetic and real-world data.
We conclude in Section~\ref{fi:sec:conclusion}.

%Link between a generalized method of moments and maximum-likelihood estimator.

\paragraph{Discrete Choice Model}
%\textbf{Discrete choice model.}
Denote by $n$ the number of items.
Luce's choice axiom implies that each item $i \in \{1, \ldots, n\}$ can be parametrized by a positive strength $\pi_i \in \mathbf{R}_{>0}$ such that $p(i \mid A) = \pi_i / \sum_{j \in A} \pi_j$ for any $A$ containing $i$.
The strengths $\bm{\pi} = [\pi_i]$ are defined up to a multiplicative factor;
%For identifiability we let $\bm{\pi} \in \Delta_n$, where $\Delta_n = \{ \bm{u} \in \mathbf{R}^n \mid u_i > 0, \sum_i u_i = 1 \}$ is the open $(n\!-\!1)$-dimensional simplex.
for identifiability, we let $\sum_i \pi_i = 1$.
An alternative parametrization of the model is given by $\theta_i = \log(\pi_i)$, in which case the model is sometimes referred to as \emph{conditional logit} \citep{mcfadden1973conditional}.

\paragraph{Markov Chain Theory}
%\textbf{Markov chain theory.}
We represent a finite, stationary, continuous-time Markov chain by a directed graph $G = (V, E)$, where $V$ is the set of states and $E$ is the set of transitions with positive rate.
If $G$ is strongly connected, the Markov chain is said to be ergodic and admits a unique \emph{stationary distribution} $\bm{\pi}$.
The global balance equations relate the transition rates $\lambda_{ij}$ to the stationary distribution as follows:
\begin{align}
\label{fi:eq:balance}
%\textstyle
\sum_{j \ne i} \pi_i \lambda_{ij} = \sum_{j \ne i} \pi_j \lambda_{ji} \quad \forall i.
\end{align}
The stationary distribution is therefore invariant to changes in the time scale, i.e., to a rescaling of the transition rates.
In the supplementary file, we briefly discuss how to find $\bm{\pi}$ given $[\lambda_{ij}]$.
