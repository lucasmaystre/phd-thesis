\newcommand{\D}[2]{\ensuremath{\mathcal{D}_{#1 \succ #2}}}

\section{Stationary points of the log-likelihood}

In this section, we briefly explain why the log-likelihood in Luce's model has a unique stationary point, that at the ML estimate.
Recall that we assume that the comparison graph $G_{\mathcal{D}}$ is strongly connected.
The log-likelihood is given by
\begin{align}
\label{fix:eq:loglik}
\log \mathcal{L}(\bm{\pi} \mid \mathcal{D}) = \sum_{\ell = 1}^d \left( \log \pi_\ell - \log{\sum_{j \in A_\ell} \pi_j} \right).
\end{align}
This function is not concave in $\bm{\pi}$; however, this does not preclude the existence of a unique stationary point.
Letting $\pi_i = e^{\theta_i}$, we write the reparametrized log-likelihood as
\begin{align*}
\log \mathcal{L}(\bm{\pi}(\bm{\theta}) \mid \mathcal{D}) = \sum_{\ell = 1}^d \left( \theta_\ell - \log{\sum_{j \in A_\ell} e^{\theta_j}} \right),
\end{align*}
which is strictly concave in $\bm{\theta}$ and therefore admits a unique stationary point, at the maximum of the function.
Denote this maximum by $\hat{\bm{\theta}}$.
The partial derivative of the log-likelihood with respect to $\pi_\ell$ is
\begin{align}
\frac{\partial \log \mathcal{L}}{\partial \pi_\ell}
  = \frac{\partial \log \mathcal{L}}{\partial \theta_i} \cdot \frac{\partial \theta_i}{\partial \pi_i}
  = \frac{\partial \log \mathcal{L}}{\partial \theta_i} \cdot \frac{1}{\pi_i}.
\end{align}
As $1/\pi_i$ is strictly positive, the partial derivative vanishes only at $\hat{\pi}_i = e^{\hat{\theta}_i}$.
In conclusion, $\hat{\bm{\pi}}$ is the unique ML estimate, as well as the only stationary point.


\section{Proofs of Theorems 1 and 2}


For any two items $i$ and $j$, recall that $\mathcal{D}_{i \succ j} \subseteq \mathcal{D}$ is the set of observations where $i$ wins over $j$.
Let $\Delta_n = \{ \bm{u} \in \mathbf{R}^n \mid u_i > 0, \sum_i u_i = 1 \}$ be the open $(n\!-\!1)$-dimensional simplex.
Recall that for $\mathcal{S} \subseteq \mathcal{D}$ and $\vpi \in \Delta_n$, we define
\begin{align}
\label{fix:eq:rate}
f(\mathcal{S}, \bm{\pi}) \doteq \sum_{A \in \mathcal{S}} \frac{1}{\sum_{i \in A} \pi_i}.
\end{align}
We will now prove the following theorem.

\begin{theorem}
\label{fix:thm:convergence}
The Markov chain with inhomogeneous transition rates $\lambda_{ji} = f(\mathcal{D}_{i \succ j}, \vpi)$ converges to the maximum-likelihood estimate $\mlpi$, for any initial distribution $\vpi^0 \in \Delta_n$.
\end{theorem}

We take a discrete-time perspective, and consider the uniformized Markov chain with (parametric) transition probabilities
\begin{align}
  P(\bm{\pi})_{ij} =
  \begin{dcases}
    \epsilon \sum_{A \in \D{j}{i}} \frac{1}{\sum_{t \in A} \pi_t}              & \text{if } j \ne i, \\
    1 - \epsilon \sum_{k \ne i}\sum_{A \in \D{k}{i}} \frac{1}{\sum_{t \in A} \pi_t}  & \text{if } j = i,
  \end{dcases}
\end{align}
where $\epsilon$ (the uniform rate parameter) is a small factor that ensures that the matrix is row-stochastic.
We say that the Markov chain is \emph{inhomogeneous} because the transition probabilities depend on the current distribution over states;
as a consequence, standard ergodic results do not apply directly.
From the development at the beginning of Section 3 of the main text, it follows that $\mlpi$ is the unique invariant distribution of the Markov chain, i.e., satisfying $\mlpi = \mlpi P(\mlpi)$.
Consider the mapping $T: \Delta_n \to \Delta_n$ defined by
\begin{align}
\label{fix:eq:mapping}
T(\vpi) = \vpi P(\vpi),
\end{align}
representing the distribution after one step of the Markov chain.
Using a contraction argument, we will show that the iteration $\vpi^{k+1} = T(\vpi^k)$ converges to a fixed point for any $\vpi^0 \in \Delta_n$.
It directly follows that the Markov chain converges to $\mlpi$ from any initial distribution.

We start with a technical lemma that characterizes the Jacobian matrix of the mapping.
We will use the notation
\begin{align}
T^k(\vpi) = \underbrace{T \circ T \circ \ldots \circ T}_{\text{$k$ times}}(\vpi)
\end{align}
for $k$ successive applications of the mapping.
We will also extend our notation for subsets of observations, and let $\D{i}{j,k} \subseteq \mathcal{D}$ be the observations where $i$ wins among a set of alternatives containing $j$ and $k$.
\begin{lemma}
\label{fix:lem:jacobian}
The Jacobian matrix of the mapping $T(\vpi)$ defined in \eqref{fix:eq:mapping} is given by
\begin{align}
T'(\vpi)_{ij} = \left[ \frac{\partial T(\vpi)}{\partial \pi_i} \right]_j =
\begin{dcases}
\epsilon \sum_{k} \sum_{A \in \D{k}{j,i}} \frac{\pi_j}{(\sum_{t \in A} \pi_t)^2}                          & \text{if } j \ne i, \\
1 - \epsilon \sum_{j \ne \ell} \sum_{k} \sum_{A \in \D{k}{j,\ell}} \frac{\pi_j}{(\sum_{t \in A} \pi_t)^2} & \text{if } j = i.
\end{dcases}
\end{align}
Furthermore, there is a finite $m \in \mathbf{N}$ such that for $S' = (T^m)'$ it holds that $\delta = \min_{i,j} S'_{ij} > 0$ and $\Vert S' \Vert_1 = 1$.
\end{lemma}

\begin{proof}
The partial derivative of $T$ with respect to $\pi_\ell$ at $j \ne \ell$ is
\begin{align}
\left[ \frac{\partial T(\vpi)}{\partial \pi_\ell} \right]_j
&= \left[ \frac{\partial \vpi}{\partial \pi_\ell} P(\vpi) \right]_j + \left[ \vpi \frac{\partial P(\vpi)}{\partial \pi_\ell} \right]_j \\
\begin{split}
&= \epsilon \sum_{A \in \D{j}{\ell}} \frac{1}{\sum_{t \in A} \pi_t}
    - \epsilon \sum_{k \ne j} \sum_{A \in \D{j}{k,\ell}} \frac{\pi_k}{(\sum_{t \in A} \pi_t)^2} \\
    & \qquad\qquad\qquad\qquad\quad + \epsilon \sum_{k \ne j} \sum_{A \in \D{k}{j,\ell}} \frac{\pi_j}{(\sum_{t \in A} \pi_t)^2}
\end{split} \label{fix:eq:before}\\
&= \epsilon \sum_{A \in \D{j}{\ell}} \frac{\pi_j}{(\sum_{t \in A} \pi_t)^2}
    + \epsilon \sum_{k \ne j} \sum_{A \in \D{k}{j,\ell}} \frac{\pi_j}{(\sum_{t \in A} \pi_t)^2} \label{fix:eq:after}\\
&= \epsilon \sum_{k} \sum_{A \in \D{k}{j,\ell}} \frac{\pi_j}{(\sum_{t \in A} \pi_t)^2}.
\end{align} 
To go from \eqref{fix:eq:before} to \eqref{fix:eq:after}, we reverse the order of summation in the subtracted term and rewrite the fraction inside the left term.
\begin{align}
 & \sum_{A \in \D{j}{\ell}} \frac{1}{\sum_{t \in A} \pi_t}
    - \sum_{k \ne j} \sum_{A \in \D{j}{k,\ell}} \frac{\pi_k}{(\sum_{t \in A} \pi_t)^2} \\
=& \sum_{A \in \D{j}{\ell}} \frac{1}{\sum_{t \in A} \pi_t}
    - \sum_{A \in \D{j}{\ell}} \sum_{k \in A, k \ne j} \frac{\pi_k}{(\sum_{t \in A} \pi_t)^2} \\
=& \sum_{A \in \D{j}{\ell}} \sum_{k \in A} \frac{\pi_k}{(\sum_{t \in A} \pi_t)^2}
    - \sum_{A \in \D{j}{\ell}} \sum_{k \in A, k \ne j} \frac{\pi_k}{(\sum_{t \in A} \pi_t)^2} \\
=& \sum_{A \in \D{j}{\ell}} \frac{\pi_j}{(\sum_{t \in A} \pi_t)^2}
\end{align}
One can find the partial derivative with respect to $\pi_\ell$ at $\ell$ by noticing that each row of the Jacobian matrix sums to one:
\begin{align}
\sum_j \left[ \frac{\partial T(\vpi)}{\partial \pi_\ell} \right]_j
&=\sum_j P(\vpi)_{\ell j} + \sum_j \sum_i \pi_i \frac{\partial P(\vpi)_{ij}}{\partial \pi_\ell} \\
&= 1 + \sum_i \pi_i \frac{\partial}{\partial \pi_\ell} \sum_j P(\vpi)_{ij} = 1.
\end{align}
The matrix is therefore row-stochastic, and $\Vert T'(\vpi) \Vert_1 = 1$.
Because transition probabilities are strictly positive on the edges of the comparison graph (which is, by assumption, strongly connected), there is a finite $m \in \mathbf{N}$ such that all entries of $T^m(\vpi)$ are lower-bounded by a strictly positive number.
It is easy to see that the Jacobian matrix $T'$ also has strictly positive entries on the edges of the comparison graph, and therefore
\begin{align}
S'(\vpi) = (T^m(\vpi))' = \prod_{i = 0}^{m-1} T'(T^i(\vpi))
\end{align}
also has its entries lower-bounded by a strictly positive number.
Furthermore, $S'(\vpi)$ is a product of stochastic matrices, hence $\Vert S'(\vpi) \Vert_1 = 1$.
\end{proof}

Now we will use the properties of the Jacobian matrix to show that $T$ is a fixed-point iteration, using a standard argument.
Our proof is inspired by the lecture notes of \citet{tresch2007convergence} and \citet{petersdorff2014fixed}.

\begin{proof}[Proof of Theorem~\ref{fix:thm:convergence}]
Using the results of Lemma~\ref{fix:lem:jacobian}, let $S(\vpi) = T^m(\vpi)$ and write $S'(\vpi)$ as
\begin{align}
S'(\vpi) = \delta 1_{n \times n} + R(\vpi),
\end{align}
where $1_{n \times n}$ is the all-ones matrix, and $\Vert R(\vpi) \Vert_1 = 1 - n\delta = c < 1$.
Now pick any $\bm{x}, \bm{y} \in \Delta_n$, and let $\tilde{S}(u) \doteq S(\bm{x} + u(\bm{x} - \bm{y}))$.
Then $\tilde{S}'(u) = S'(\bm{x} + u(\bm{y} - \bm{x}))(\bm{y} - \bm{x})$, and
\begin{align}
S(\bm{y}) - S(\bm{x}) = \tilde{S}(1) - \tilde{S}(0) = \int_0^1 \tilde{S}'(u) du = \int_0^1 S'(\bm{x} + u(\bm{y} - \bm{x}))(\bm{y} - \bm{x}) du
\end{align}
As $S'$ is continuous, we have
\begin{align}
\Vert S(\bm{y}) - S(\bm{x}) \Vert_1 &\le \int_0^1 \Vert S'(\bm{x} + u(\bm{y} - \bm{x}))(\bm{y} - \bm{x}) \Vert_1 du \\
&=   \int_0^1 \Vert \underbrace{\delta 1_{n \times n}(\bm{y} - \bm{x})}_{= 0} + R(\bm{x} + u(\bm{y} - \bm{x}))(\bm{y} - \bm{x}) \Vert_1 du \\
&\le \int_0^1 \underbrace{\Vert R(\bm{x} + u(\bm{y} - \bm{x})) \Vert_2}_{\le c} \Vert \bm{y} - \bm{x} \Vert_1 du \\
&\le c \Vert \bm{y} - \bm{x} \Vert_1
\end{align}
Therefore, by the contraction mapping principle, the sequence of iterates $\vpi^{k+1} = T^m(\vpi^k)$ converges to $\mlpi$.
Finally, we observe that for any $\vpi \in \Delta_n$, the vectors $\vpi, T(\vpi), T^2(\vpi), \ldots$ occur in one of the sequences
\begin{align}
\left( S^k(T^r(\vpi)) \right)_{k \in \mathbf{N}_0}, \quad r \in \{ 0, \ldots, m\!-\!1 \}.
\end{align}
All sequences converge to $\mlpi$, and therefore
\begin{align}
\lim_{k \to \infty} T^k(\vpi) = \mlpi.
\end{align}
\end{proof}

\begin{theorem}
\label{fix:thm:consistency}
Let $\mathcal{A} = \{ A_\ell \}$ be a collection of sets of alternatives such that for any partition of $\mathcal{A}$ into two non-empty sets $S$ and $T$, $\left( \cup_{A \in S} A \right) \cap \left( \cup_{A \in T} A \right) \ne \varnothing$.
Let $d_\ell$ be the number of choices observed over alternatives $A_\ell$.
Then $\rcpi \to \bm{\pi}^*$ as $d_\ell \to \infty \ \forall \ell$.
\end{theorem}

\begin{proof}
Let $d \to \infty$ be a shorthand for $d_\ell \to \infty \ \forall \ell$.
The condition on $\mathcal{A}$ is equivalent to stating that the hypergraph $H = (V, \mathcal{A})$ with $V = \{1, \ldots, n\}$ is connected.
First, we show that asymptotically, the graph $G_\mathcal{D} = (V, E)$ is connected.
For a given set of alternatives $A_\ell$, let $i, j \in A_\ell$.
The probability that $(j, i) \in E$ is
\begin{align}
1 - \left(1 - \frac{\pi_i}{\sum_{t \in A_\ell} \pi_t} \right)^{d_\ell}
> 1 - (1 - \pi_i)^{d_\ell}
\xrightarrow{d_\ell \to \infty} 1,
\end{align}
where we use the fact that $\pi_i > 0 \ \forall i$.
Therefore, asymptotically, every alternative set $A_\ell$ forms a clique in $G_\mathcal{D}$.
By assumption of connectivity on the hypergraph $H$, $G_{\mathcal{D}}$ is strongly connected.

Now that we know that the Markov chain is asymptotically ergodic, we will show that the stationary distribution matches the true model parameters.
Let $C_\ell^s$ be a random variable denoting the item chosen in the $s$-th observation over alternatives $A_\ell$.
By the law of large numbers, for any item $i \in A_\ell$
\begin{align}
\label{fix:eq:lln}
\lim_{d_\ell \to \infty} \frac{1}{d_\ell} \sum_{s = 1}^{d_\ell} 1\{C_\ell^s = i\} = \frac{\pi^*_i}{\sum_{t \in A_\ell} \pi^*_t}.
\end{align}
Now consider two items $i$ and $j$.
If they have never been compared, $\lambda_{ij} = \lambda_{ji} = 0$.
Otherwise, suppose that they have been compared in alternative sets whose indices are in $B = \{ \ell \mid i, j \in A_\ell \}$
Let $\mathbf{1}\{X\}$ be the indicator variable for event $X$.
By construction of the transition rates in \LSR{}, we have that
\begin{align}
\label{fix:eq:ratio}
\frac{\lambda_{ij}}{\lambda_{ji}}
= \frac{\sum_{\ell \in B} \sum_{s = 1}^{d_\ell} \mathbf{1}\{C_\ell^s = j\} \ n / |A_\ell|}
       {\sum_{\ell \in B} \sum_{s = 1}^{d_\ell} \mathbf{1}\{C_\ell^s = i\} \ n / |A_\ell|}.
\end{align}
From \eqref{fix:eq:lln} it follows that
\begin{align}
\lim_{d \to \infty}\frac{\lambda_{ij}}{\lambda_{ji}}
& = \frac{\sum_{\ell \in B} (\pi^*_j / \sum_{t \in A_\ell} \pi^*_t) \ n / |A_\ell|}
         {\sum_{\ell \in B} (\pi^*_i / \sum_{t \in A_\ell} \pi^*_t) \ n / |A_\ell|} \\
& = \frac{\pi^*_j }{\pi^*_i}
    \cdot \frac{\sum_{\ell \in B}(1 / \sum_{t \in A_\ell} \pi^*_t) \ n / |A_\ell|}
               {\sum_{\ell \in B}(1 / \sum_{t \in A_\ell} \pi^*_t) \ n / |A_\ell|}
  = \frac{\pi^*_j}{\pi^*_i}.
\end{align}
Therefore, when $d \to \infty$,
\begin{align}
\sum_{j \ne i} \pi^*_i \lambda_{ij} = \sum_{j \ne i} \pi^*_i \left( \frac{\pi^*_j}{\pi^*_i} \lambda_{ji} \right)
                                    = \sum_{j \ne i} \pi^*_j \lambda_{ji}  \quad \forall i.
\end{align}
It is easy to recognize the global balance equations, and it follows that $\bm{\pi}^*$ is the stationary distribution of the asymptotical Markov chain.
\end{proof}


\section{Bound on error rate of ML estimate}

We use the analytical framework of \citet{negahban2017rank} to bound the error rate of the ML estimator in the case where
\begin{enuminline}
\item the data is in the form of pairwise comparisons and
\item for each pair under comparison, we observe exactly $k$ outcomes.
\end{enuminline}

Let $G = (V, E)$ be an undirected graph where $V = \{ 1, \ldots, n \}$ and $(i, j) \in E$ if $i$ and $j$ have been compared.
Let $d_{\min}$ and $d_{\max}$ be the minimum and maximum degree of a node in $G$, respectively.
Let $\gamma$ be the spectral gap of a simple random walk on $G$;
intuitively, the larger the spectral gap is, the faster the convergence to the stationary distribution is.
For each $(i, j) \in E$ we observe $k$ comparisons generated from ground truth parameters $\bm{\pi}^*$.
Let $A_{ji}$ denote the number of times $i$ wins against $j$ and $a_{ji} = A_{ji} / k$ the ratio of wins of $i$ over $j$.
We say that an event $X$ occurs with high probability if $\mathbf{P}(X) \ge 1 - c / n^\alpha$ for $c, \alpha$ fixed.

\begin{theorem}
\label{fix:thm:mlbound}
For $k \ge 4C^2 (1 + (b^6 \kappa^2 / (d_{\max} \gamma^2)) \log n)$, the error on the ML estimate $\mlpi$ satisfies w.h.p.
\begin{align}
\frac{\Vert \mlpi - \bm{\pi}^* \Vert_2}{\Vert \bm{\pi}^* \Vert_2} < C \frac{b^{7/2} \kappa}{\gamma} \sqrt{\frac{\log n}{kd_{\max}}},
\end{align}
where $C$ is a constant, $b = \max_{i,j} \pi^*_i / \pi^*_j$ and $\kappa = d_{\max} / d_{\min}$.
\end{theorem}

\begin{proof}
The ML estimate can be interpreted as the stationary distribution of the discrete-time Markov chain
\begin{align}
\label{fix:eq:mlchain}
  \widehat{P}_{ij} =
  \begin{dcases}
    \epsilon \frac{a_{ij}}{\hat{\pi}_i + \hat{\pi}_j}                    & \text{if } i \ne j, \\
    1 - \epsilon \sum_{l \ne i} \frac{a_{il}}{\hat{\pi}_i + \hat{\pi}_l} & \text{if } i = j.
  \end{dcases}
\end{align}
The factor $\epsilon = \hat{\pi}_{\min} / d_{\max}$ ensures that $\widehat{P}$ is stochastic.
Given this matrix, it is straightforward to analyze the $\mlpi$ by using the methods developed for Rank Centrality (RC);
the proof essentially follows that of Theorem~1 of \citet{negahban2017rank}.
Let $P^*$ be the ideal Markov chain, when  $a_{ij} = \pi^*_j / (\pi^*_i + \pi^*_j)$, i.e., the ratios are noiseless.
The key observation is to note that the stationary distribution of $P^*$ is $\bm{\pi}^*$, the true model parameters.
By bounding $\Vert \widehat{P} - P^* \Vert_2$ and $1 - \lambda_{\max}(P^*)$, we can bound the error on the stationary distribution of $P^*$.
For the former, a straightforward application of the proof in the RC case suffices.
For the latter, in the application of the comparison theorem, the lower bound on $\min_{i,j} \pi^*_i P^*_{ij}$ changes by a factor of $1/(2b)$.
This is due to the additional factor $\hat{\pi}_{\min} / (\hat{\pi}_i + \hat{\pi}_j)$ in the off-diagonal entries of $P^*$.
\end{proof}

If the graph of comparisons $G$ is an expander, then $\gamma = O(1)$.
Furthermore, if $d_{\max} \propto d_{\min}$, then $\kappa = O(1)$.
A realization of the $G(n, p)$ random graph satisfies these two constraints with high probability as long as $p = \omega(\log n / n)$.
It follows that if $\omega(n \log n)$ comparison pairs are chosen uniformly at random and $k = O(1)$ outcomes are observed for each pair, the error goes to zero as $n$ increases.

\citet{hajek2014minimax} recently proved a more general version of our result, using a different analytical technique.
Their bound is qualitatively similar, but also applies to multiway rankings and heterogeneous number of comparisons.

\section{Derivation for the Rao--Kupper model}

We consider a model that was proposed by Rao and Kupper in 1967 \citep{rao1967ties}.
This model extends the Bradley--Terry model in that a comparison between two items can result in a tie.
Letting $\alpha \in [1, \infty)$, the probabilities of $i$ winning over and tying with $j$, respectively, are given as follows.
\begin{align*}
p(i \succ j) &= \frac{\pi_i}{\pi_i + \alpha \pi_j}, \\
p(i \leftrightarrow j) &= \frac{\pi_i \pi_j(\alpha^2 - 1)}{(\pi_i + \alpha\pi_j)(\alpha \pi_i + \pi_j)}.
\end{align*}
This model is useful for e.g., chess, where a significant fraction of comparison outcomes do not result in either a win or a loss.

We assume that the parameter $\alpha$ is fixed, and derive an expression of the ML estimate $\mlpi$.
Let $A_{ji}$ be the number of times $i$ wins over $j$, and $T_{ij} = T_{ji}$ be the number of ties between $i$ and $j$.
The log-likelihood can be written as
\begin{align}
\log \mathcal{L} &=
  \sum_i \sum_{j \ne i}
  A_{ji} \left( \log(\pi_i) - \log(\pi_i + \alpha \pi_j) \right) \\
    &+ \sum_i \sum_{j > i} \nonumber
    T_{ij} ( \log(\pi_i) + \log(\pi_j) + \log(\alpha^2 - 1) \\
    & \qquad \qquad {} - \log(\pi_i + \alpha \pi_j) - \log(\alpha \pi_i + \pi_j) ). \nonumber
\end{align}
The log-likelihood function is strictly concave and the model admits a unique ML estimate $\mlpi$.
The optimality condition $\nabla_{\mlpi} \log \mathcal{L} = 0$ implies
\begin{align}
  \frac{\partial \log \mathcal{L}}{\partial \hat{\pi}_i}
  = &\sum_{j \ne i} A_{ji} \left( \frac{1}{\hat{\pi}_i} - \frac{1}{\hat{\pi}_i + \alpha \hat{\pi}_j} \right)
    - A_{ij} \frac{\alpha}{\alpha \hat{\pi}_i + \hat{\pi}_j} \\
  & \qquad {} + T_{ij} \left( \frac{1}{\hat{\pi}_i} - \frac{1}{\hat{\pi}_i + \alpha \hat{\pi}_j} - \frac{\alpha}{\alpha \hat{\pi}_i + \hat{\pi}_j}\right) = 0 \\
  \iff & \sum_{j \ne i} A_{ji} \frac{\alpha \hat{\pi}_j}{\hat{\pi}_i + \alpha \hat{\pi}_j}
    - A_{ij} \frac{\alpha \hat{\pi}_i}{\alpha \hat{\pi}_i + \hat{\pi}_j} \\
  & \qquad {} + T_{ij} \frac{\alpha \hat{\pi}_j^2 - \alpha \hat{\pi}_i^2}{(\hat{\pi}_i + \alpha \hat{\pi}_j)(\alpha \hat{\pi}_i + \hat{\pi}_j)} = 0 \\
  \iff & \sum_{j \ne i} \frac{A_{ji} + T_{ji}\tfrac{\hat{\pi}_j}{\alpha \hat{\pi}_i + \hat{\pi}_j}}{\hat{\pi}_i + \alpha \hat{\pi}_j} \hat{\pi}_j
    - \frac{A_{ij} + T_{ij}\tfrac{\hat{\pi}_i}{\hat{\pi}_i + \alpha \hat{\pi}_j}}{\alpha \hat{\pi}_i + \hat{\pi}_j} \hat{\pi}_i = 0.
\end{align}
Therefore, the ML estimate is the stationary distribution of a Markov chain with transition rates
\begin{align}
\lambda_{ij} = \frac{A_{ij} + T_{ij}\tfrac{\hat{\pi}_i}{\hat{\pi}_i + \alpha \hat{\pi}_j}}{\alpha \hat{\pi}_i + \hat{\pi}_j}.
\end{align}
The extension of \LSR{} and \ILSR{} to the Rao--Kupper model given these transition rates is straightforward.



\section{Finding the stationary distribution}

A set of transition rates $[\lambda_{ij}]$ that satisfy the strong connectivity assumption yields a unique stationary distribution $\bm{\pi}$.
In practice, finding this stationary distribution can be implemented in various ways.
We distinguish implementations based on whether they consider a continuous-time or a discrete-time perspective on Markov chains.

\paragraph{Continuous-time perspective.}
We consider the infinitesimal generator matrix $Q$, where $Q_{ij} \doteq \lambda_{ij}$ and $Q_{ii} \doteq - \sum_{j} \lambda_{ij}$.
The stationary distribution satisfies $\bm{\pi} Q = 0$; this is essentially a matrix formulation of the global balance equations.
Therefore, one approach to finding the steady-state distribution is to compute the rank-$1$ left nullspace of $Q$.
This can be done e.g., by LU decomposition, a basic linear-algebra primitive.
In the dense case, the running time of a typical implementation is $O(n^3)$, but highly optimized parallel implementations such as that provided by LAPACK \citep{anderson1999lapack} are commonly available.
In the sparse case, LU decomposition can be done significantly faster using adapted algorithms, such as that of \citet{demmel1999supernodal}.

\paragraph{Discrete-time perspective.}
Let $\epsilon < 1 / \max_i |Q_{ii}|$, then $P = I + \epsilon Q$ is the transition matrix of a discrete-time Markov chain that satisfies $\bm{\pi} P = \bm{\pi}$.
In this case, finding the steady-state distribution is equivalent to finding the left eigenvector associated to the leading eigenvalue of the transition matrix $P$.
This is also a well-studied linear algebra problem for which plenty of efficient, off-the-shelf algorithms exist.
For example, power iteration methods can find the eigenvector in a few (sparse) matrix multiplications.
Beyond these well-known algorithms, the recently proposed randomized approach of \citet{halko2011finding} enables us to scale to truly large problem sizes ($n$ is $O(10^6)$ or more.)

For our experiments, we have implemented \LSR{} and \ILSR{} using a dense LU factorization of the generator matrix.
The Python code, which relies on the \texttt{numpy} and \texttt{scipy} libraries\footnote{
See: \url{http://www.scipy.org/}.
}, is displayed in Figure~\ref{fix:lst:implementation}

%\lstset{
%  language=Python,
%  basicstyle=\ttfamily\footnotesize,         % Teletype font.
%  showstringspaces=false,       % Don't put underscores in place of spaces.
%  numberstyle=\ttfamily, % Numbers in gray (+ teletype font).
%  numbers=left,                 % Show line numbers on the left.
%  numbersep=5pt,                % Give some space to those poor line numbers.
%  xleftmargin=15pt,             % But don't put them in the margin either.
%  %columns=fixed,               % No idea what this is for.
%}
\begin{figure}
\begin{lstlisting}
import numpy as np
import scipy.linalg as spl

def weighted_lsr(n, rankings, weights):
    chain = np.zeros((n, n), dtype=float)
    for ranking in rankings:
        sum_weights = sum(weights[x] for x in ranking)
        for i, winner in enumerate(ranking):
            val = 1.0 / sum_weights
            for loser in ranking[i+1:]:
                chain[loser, winner] += val
            sum_weights -= weights[winner]
    chain -= np.diag(chain.sum(axis=1))
    return statdist(chain)

def statdist(chain):
    lu, piv = spl.lu_factor(generator.T)
    res = spl.solve_triangular(lu[:-1,:-1], -lu[:-1,-1])
    res = np.append(res, 1.0)
    return res / res.sum()
\end{lstlisting}
\caption{
Python implementation of one iteration of \ILSR{}.
}
\label{fix:lst:implementation}
\end{figure}

\section{Experimental procedure}

We give a few additional details on the procedure that we followed for the experiments of Section~4 in the main paper.
All experiments were run on a machine with a quad-core 2.0 GHz Haswell processor, and 16GB of RAM, running Mac OS X 10.9.
For \LSR{} and \ILSR{}, we used a slightly adapted version the code presented in Figure~\ref{fix:lst:implementation}.
We implemented the Rank Centrality (RC), GMM-F \citep{azari2013generalized}, and MM \citep{hunter2004mm} algorithms in Python.
For Newton-Raphson, we implemented our choice model on top of the popular \texttt{statsmodels} Python library\footnote{
See: \url{http://statsmodels.sourceforge.net/}
} that provides a Newton-Raphson solver.
For completeness, the Python source code containing all the functions we used is provided as a separate file in the supplementary material.
We have compared our implementation of the MM algorithm to that of \citeauthor{hunter2004mm} written in Matlab\footnote{
See: \url{http://sites.stat.psu.edu/~dhunter/code/btmatlab/}
}, and observed that ours has comparable running time.

For the chess dataset, we use the Rao--Kupper model and set the parameter $\alpha = \sqrt{2}$.
Note that this parameter could also be estimated from the data, however in our experiments we focus on the performance of algorithms for estimating $\mlpi$.
