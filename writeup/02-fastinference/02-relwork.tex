%%%%%%%%%%%%%%%%%%%%%%%%%%%%%%%%
\section{Related Work}
\label{fi:sec:relwork}

Spectral methods applied to ranking and scoring items from noisy choices have a long-standing history.
To the best of our knowledge, \citet{saaty1980analytic} is the first to suggest using the leading eigenvector of a matrix of inconsistent pairwise judgments to score alternatives.
Two decades later, \citet{page1998pagerank} developed PageRank, an algorithm that ranks Web pages according to the stationary distribution of a random walk on the hyperlink graph.
In the same vein, \citet{dwork2001rank} proposed several variants of Markov chains for aggregating heterogeneous rankings.
The idea is to construct a random walk that is biased towards high-ranked items, and use the ranking induced by the stationary distribution.
More recently, \citet{negahban2012iterative} presented Rank Centrality, an algorithm for aggregating pairwise comparisons close in spirit to that of \citep{dwork2001rank}.
When the data is generated under the Bradley--Terry model, this algorithm asymptotically recovers model parameters with only $\LittleOmega{N \log N}$ pairwise comparisons.
For the more general case of rankings under the Plackett--Luce model, \citet{azari2013generalized} propose to break rankings into pairwise comparisons and to apply an algorithm similar to Rank Centrality.
They show that the resulting estimator is statistically consistent.
%Lastly, we mention \citep{fogel2014serialrank} who take a slightly different approach to spectral ranking, based on seriation, w
Interestingly, many of these spectral algorithms can be related to the method of moments, a broadly applicable alternative to maximum-likelihood estimation.
%Spectral methods / method of moments successfuly applied to other problems \citet{anandkumar2012spectral}

%Connection between ML and stationary distribution of a Markov chain: \citet{kumar2015inverting}

The history of algorithms for maximum-likelihood inference under Luce's model goes back even further.
In the special case of pairwise comparisons, the same iterative algorithm was independently discovered by \citet{zermelo1928berechnung}, \citet{ford1957solution} and \citet{dykstra1960rank}.
Much later, this algorithm was explained by \citet{hunter2004mm} as an instance of minorization-maximization (MM) algorithm, and extended to the more general choice model.
Today, Hunter's MM algorithm is the \emph{de facto} standard for ML inference in Luce's model.
As the likelihood can be written as a concave function, off-the-shelf optimization procedures such as the Newton-Raphson method can also be used, although they have been been reported to be slower and less practical \citep{hunter2004mm}.
Recently, \citet{kumar2015inverting} looked at the problem of finding the transition matrix of a Markov chain, given its stationary distribution.
The problem of inferring Luce's model parameters from data can be reformulated in their framework, and the MLE is the solution to the inversion of the stationary distribution.
Their work stands out as the first to link ML inference to Markov chains, albeit very differently from the way presented in this chapter.

Beyond algorithms, properties of the maximum-likelihood estimator in this model were studied extensively.
%\citet{simons1999asymptotics} show that the estimator is asymptotically normal in the finite sample case.
\citet{hajek2014minimax} consider the Plackett--Luce model for $K$-way rankings.
They give an upper bound to the estimation error and show that the MLE is minimax-optimal.
In summary, they show that only $\LittleOmega{N/K \log N}$ samples are enough to drive the mean-square error down to zero, as $N$ increases.
\citet{rajkumar2014statistical} consider the Bradley--Terry model for pairwise comparisons.
They show that the ML estimator is able to recover the correct ranking, even when the data is generated as per another model, e.g., Thurstone's \citep{thurstone1927method}, as long as a so-called \emph{low-noise} condition is satisfied.
%These results expose the fact the ML estimate enjoys numerous desirable qualities, and makes are relevant and useful to practitioners.
We also mention that as an alternative to likelihood maximization, Bayesian inference has also been proposed.
\citet{caron2012efficient} present a Gibbs sampler, and \citet{guiver2009bayesian} propose an approximate inference algorithm based on expectation propagation.

In this chapter, we provide a unifying perspective on recent advances in spectral algorithms \citep{negahban2012iterative, azari2013generalized} from a maximum-likelihood estimation perspective.
It turns out that this perspective enables us to make contributions on both sides:
On the one hand, we develop an improved and more general spectral ranking algorithm, and on the other hand, we propose a faster procedure for ML inference by using this algorithm iteratively.
