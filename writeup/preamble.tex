%%%%%%%%%%%%%%%%%%%%%%%%%%%%%%%%%%%%%%%%%%%%%%
%
%		Thesis Settings
%
%		EDOC Template
%		2011
%
%%%%%%%%%%%%%%%%%%%%%%%%%%%%%%%%%%%%%%%%%%%%%%
\documentclass[a4paper,11pt,fleqn]{book}

\usepackage[T1]{fontenc}
\usepackage[utf8]{inputenc}
\usepackage{natbib}  % Has to be loaded before babel.
\usepackage[french,english]{babel}


%%%%%%%%%%%%%%%%%%%%%%%%%%%%%%%%%%%%%%%%%%%%%%%
%% EDOC THESIS TEMPLATE: Variant 1.0 -> Latin modern, large text width&height
%%%%%%%%%%%%%%%%%%%%%%%%%%%%%%%%%%%%%%%%%%%%%%%
%\usepackage{lmodern}
%\usepackage[a4paper,top=22mm,bottom=28mm,inner=35mm,outer=25mm]{geometry}
%%%%%%%%%%%%%%%%%%%%%%%%%%%%%%%%%%%%%%%%%%%%%%%

%%%%%%%%%%%%%%%%%%%%%%%%%%%%%%%%%%%%%%%%%%%%%%
% EDOC THESIS TEMPLATE: Variant 2.0 -> Utopia, Gabarrit A (lighter pages)
%%%%%%%%%%%%%%%%%%%%%%%%%%%%%%%%%%%%%%%%%%%%%%
\usepackage{lmodern}
\setlength{\textwidth}{146.8mm} % = 210mm - 37mm - 26.2mm
\setlength{\oddsidemargin}{11.6mm} % 37mm - 1in (from hoffset)
\setlength{\evensidemargin}{0.8mm} % = 26.2mm - 1in (from hoffset)
\setlength{\topmargin}{-2.2mm} % = 0mm -1in + 23.2mm 
\setlength{\textheight}{221.9mm} % = 297mm -29.5mm -31.6mm - 14mm (12 to accomodate footline with pagenumber)
\setlength{\headheight}{14pt}
%%%%%%%%%%%%%%%%%%%%%%%%%%%%%%%%%%%%%%%%%%%%%%


\setlength{\parindent}{0pt}

\usepackage{setspace} % increase interline spacing slightly
\setstretch{1.1}

\makeatletter
\setlength{\@fptop}{0pt}  % for aligning all floating figures/tables etc... to the top margin
\makeatother


% No need to pass `pdftex` option to graphicx:
% https://tex.stackexchange.com/a/82158/79036
\usepackage{graphicx,xcolor}
\graphicspath{{figs/}}

\usepackage{booktabs}
\usepackage[babel,final]{microtype}
\usepackage{url}
\usepackage[final]{pdfpages}

\usepackage{fancyhdr}
\renewcommand{\sectionmark}[1]{\markright{\thesection\ #1}}
\pagestyle{fancy}
	\fancyhf{}
	\renewcommand{\headrulewidth}{0.4pt}
	\renewcommand{\footrulewidth}{0pt}
	\fancyhead[OR]{\bfseries \nouppercase{\rightmark}}
	\fancyhead[EL]{\bfseries \nouppercase{\leftmark}}
	\fancyfoot[EL,OR]{\thepage}
\fancypagestyle{plain}{
	\fancyhf{}
	\renewcommand{\headrulewidth}{0pt}
	\renewcommand{\footrulewidth}{0pt}
	\fancyfoot[EL,OR]{\thepage}}
\fancypagestyle{addpagenumbersforpdfimports}{
	\fancyhead{}
	\renewcommand{\headrulewidth}{0pt}
	\fancyfoot{}
	\fancyfoot[RO,LE]{\thepage}
}

\usepackage{listings}
\lstset{
  language=Python,
  basicstyle=\scriptsize\ttfamily,
  showstringspaces=false,           % Don't put underscores in place of spaces.
  numbers=left,                     % Show line numbers on the left.
  numberstyle=\ttfamily,
  numbersep=10pt,                   % Give some space to those poor line numbers.
  breaklines=true,
  breakautoindent=true,
  breakindent=10pt}



\makeatletter
\def\cleardoublepage{\clearpage\if@twoside \ifodd\c@page\else
    \hbox{}
    \thispagestyle{empty}
    \newpage
    \if@twocolumn\hbox{}\newpage\fi\fi\fi}
\makeatother \clearpage{\pagestyle{plain}\cleardoublepage}


%%%%% CHAPTER HEADER %%%%
\usepackage{color}
\usepackage{tikz}
\usepackage[explicit]{titlesec}
\newcommand*\chapterlabel{}
%\renewcommand{\thechapter}{\Roman{chapter}}
\titleformat{\chapter}[display]  % type (section,chapter,etc...) to vary,  shape (eg display-type)
	{\normalfont\bfseries\Huge} % format of the chapter
	{\gdef\chapterlabel{\thechapter\ }}     % the label 
 	{0pt} % separation between label and chapter-title
 	  {\begin{tikzpicture}[remember picture,overlay]
    \node[yshift=-8cm] at (current page.north west)
      {\begin{tikzpicture}[remember picture, overlay]
        \draw[fill=black] (0,0) rectangle(35.5mm,15mm);
        \node[anchor=north east,yshift=-7.2cm,xshift=34mm,minimum height=30mm,inner sep=0mm] at (current page.north west)
        {\parbox[top][30mm][t]{15mm}{\raggedleft $\phantom{\textrm{l}}$\color{white}\chapterlabel}};  %the black l is just to get better base-line alingement
        \node[anchor=north west,yshift=-7.2cm,xshift=37mm,text width=\textwidth,minimum height=30mm,inner sep=0mm] at (current page.north west)
              {\parbox[top][30mm][t]{\textwidth}{\color{black}#1}};
       \end{tikzpicture}
      };
   \end{tikzpicture}
   \gdef\chapterlabel{}
  } % code before the title body

\titlespacing*{\chapter}{0pt}{50pt}{30pt}
\titlespacing*{\section}{0pt}{13.2pt}{*0}  % 13.2pt is line spacing for a text with 11pt font size
\titlespacing*{\subsection}{0pt}{13.2pt}{*0}
\titlespacing*{\subsubsection}{0pt}{13.2pt}{*0}

\newcounter{myparts}
\newcommand*\partlabel{}
\titleformat{\part}[display]  % type (section,chapter,etc...) to vary,  shape (eg display-type)
	{\normalfont\bfseries\Huge} % format of the part
	{\gdef\partlabel{\thepart\ }}     % the label 
 	{0pt} % separation between label and part-title
 	  {\setlength{\unitlength}{20mm}
	  \addtocounter{myparts}{1}
	  \begin{tikzpicture}[remember picture,overlay]
    \node[anchor=north west,xshift=-65mm,yshift=-6.9cm-\value{myparts}*20mm] at (current page.north east) % for unknown reasons: 3mm missing -> 65 instead of 62
      {\begin{tikzpicture}[remember picture, overlay]
        \draw[fill=black] (0,0) rectangle(62mm,20mm);   % -\value{myparts}\unitlength
        \node[anchor=north west,yshift=-6.1cm-\value{myparts}*20mm,xshift=-60.5mm,minimum height=30mm,inner sep=0mm] at (current page.north east)
        {\parbox[top][30mm][t]{55mm}{\raggedright \color{white}Part \partlabel $\phantom{\textrm{l}}$}};  %the phantom l is just to get better base-line alingement
        \node[anchor=north east,yshift=-6.1cm-\value{myparts}*20mm,xshift=-63.5mm,text width=\textwidth,minimum height=30mm,inner sep=0mm] at (current page.north east)
              {\parbox[top][30mm][t]{\textwidth}{\raggedleft \color{black}#1}};
       \end{tikzpicture}
      };
   \end{tikzpicture}
   \gdef\partlabel{}
  } % code before the title body


\usepackage{import}  % See <https://ctan.org/pkg/import>.
\usepackage{amsmath,amsthm,amssymb,bm,mathtools}
\usepackage{enumitem}
\usepackage[detect-weight=true]{siunitx}
\usepackage{xifthen}  % Enables if-then-else statements in command defs.
\usepackage{caption}  % TODO: Do we need the option `style=base`?
\usepackage{subcaption}

% `hyperref` should be loaded close to the end:
% - http://mirror.switch.ch/ftp/mirror/tex/macros/latex/contrib/hyperref/README.pdf
% - https://tex.stackexchange.com/questions/1863/
\usepackage[backref=page]{hyperref}
\hypersetup{pdfborder={0 0 0},
	colorlinks=true,
	linkcolor=black,
	citecolor=black,
	urlcolor=black}

% `algorithm` should be loaded after `hyperref`.
\usepackage[chapter]{algorithm}
\usepackage{algpseudocode}

% Custom backref, taken from Vincent.
\renewcommand*{\backref}[1]{}
\renewcommand*{\backrefalt}[4]{{\footnotesize [%
    \ifcase #1 Not cited%
          \or Cited on page~#2%
          \else Cited on pages #2%
    \fi%
    ]}}

\bibliographystyle{abbrvnat}

\newlist{enuminline}{enumerate*}{1}
\setlist[enuminline,1]{label=(\itshape\alph*\upshape)}

% Removes double spacing after end of sentence.
% See: http://practicaltypography.com/one-space-between-sentences.html.
\frenchspacing

\theoremstyle{plain}
\newtheorem{theorem}{Theorem}[chapter]
\newtheorem{corollary}[theorem]{Corollary}
\newtheorem{lemma}[theorem]{Lemma}
\newtheorem{proposition}[theorem]{Proposition}

\theoremstyle{definition}
\newtheorem*{definition}{Definition}

% Probability.
\newcommand{\Prob}[1]{\ensuremath{\mathbf{P}\left[ #1 \right]}}
\newcommand{\Exp}[2][]{\ensuremath{%
\ifthenelse{\isempty{#1}}{\mathbf{E}\left[ #2 \right]}{\mathbf{E}_{#1}\left[ #2 \right]}}}
\newcommand{\Cov}[2]{\ensuremath{\mathbf{Cov}\left[ #1, #2 \right]}}
\newcommand{\Var}[1]{\ensuremath{\mathbf{Var}\left[ #1 \right]}}
\newcommand{\Indic}[1]{\ensuremath{\mathbf{1} \left \{ #1 \right \} }}

% Distributions.
\newcommand{\DExp}[1]{\ensuremath{\mathrm{Exp}( #1 )}}
\newcommand{\DGamma}[1]{\ensuremath{\mathrm{Gamma}( #1 )}}
\newcommand{\DGumbel}[1]{\ensuremath{\mathrm{Gumbel}( #1 )}}
\newcommand{\DBeta}[1]{\ensuremath{\mathrm{Beta}( #1 )}}
\newcommand{\DUnif}[1]{\ensuremath{\mathrm{U}( #1 )}}
\newcommand{\DNorm}[1]{\ensuremath{\mathrm{N}( #1 )}}

% Asymptotics
\newcommand{\BigO}[2][]{\ensuremath{%
\ifthenelse{\isempty{#1}}{O( #2 )}{O_{ #1 }( #2 )}}}
\newcommand{\LittleO}[1]{\ensuremath{o(#1)}}
\newcommand{\BigOmega}[1]{\ensuremath{\Omega(#1)}}
\newcommand{\LittleOmega}[1]{\ensuremath{\omega(#1)}}

% Paired delimiters.
\DeclarePairedDelimiter{\Ceil}{\lceil}{\rceil}
\DeclarePairedDelimiter{\Abs}{\lvert}{\rvert}
\DeclarePairedDelimiter{\Norm}{\lVert}{\rVert}

% Algorithms.
\algnewcommand{\OneLineFor}[1]{\State \algorithmicfor\ #1\ \algorithmicdo}
\algnewcommand{\OneLineIf}[1]{\State \algorithmicif\ #1\ \algorithmicthen}

% Various
\newcommand{\Tr}{\top}
\newcommand{\Diag}[1]{\ensuremath{\text{diag} \left( #1 \right) }}
\newcommand{\IntInt}[2]{\ensuremath{\{#1 \; .. \; #2 \}}}
\newcommand{\Disp}[2][]{\ensuremath{%
\ifthenelse{\isempty{#1}}{\Delta( #2 )}{\Delta_{#1}( #2 )}}}
\newcommand{\Id}{\ensuremath{\mathrm{id}}}
\DeclareMathOperator{\BT}{BT}
\DeclareMathOperator{\KL}{KL}
\DeclareMathOperator*{\Argmin}{arg\,min}
\DeclareMathOperator*{\Argmax}{arg\,max}


\hyphenation{wahrscheinlichkeits-rechnung PageRank ChoiceRank}
