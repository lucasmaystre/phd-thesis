\cleardoublepage
\chapter*{Mathematical Notation}
%\markboth{Abstract}{Abstract}  % TODO: Do we need this if the abstract takes several pages?
\addcontentsline{toc}{chapter}{Mathematical Notation} % adds an entry to the table of contents

\begingroup
\renewcommand*{\arraystretch}{1.5}
\begin{tabularx}{\linewidth}{lX}
  Symbol        & Description \\
  \midrule
  $x$
    & normal face denote scalar value \\
  $\bm{x} = [x_i]$
    & boldface lowercase to denote a column vector \\
  $\bm{X} = [x_{ij}]$
    & boldface uppercase to denote a column vector \\
  $\mathcal{X}$
    & Calligraphic uppercase to denote a set \\
  $\mathbf{R}$ $\mathbf{R}_{>0}$ $\mathbf{N}$
    & the real numbers,the positive reals, the natural numbers. \\
  $[N]$
    & Set of consecutive natural numbers $\{ 1, \ldots, N \}$ \\
  $i \succ j$
    & event ``$i$ wins over $j$'' in a pairwise comparison \\
  $i \succeq \mathcal{A}$
    & event ``$i$ is chosen among alternatives $\mathcal{A}$'' \\
  $\Prob{\mathcal{A}}$
    & the probability of event $\mathcal{A}$ \\
  $\Indic{\mathcal{A}}$
    & Indicator variable of event $\mathcal{A}$ \\
  $\Exp{x}$
    & Expectation of the random variable $x$ \\
  $\Var{x}$
    & Variance of the random variable $x$ \\
  $\Cov{x}{y}$
    & Covariance of the random variables $x$ and $y$. \\
  $\BigO{f}$
    & $g = \BigO{f} \iff \limsup_{x \to \infty} \Abs{g(x)} / f(x) < \infty$. \\
  $\LittleO{f}$
    & $g = \LittleO{f} \iff \lim_{x \to \infty} g(x) / f(x) = 0$. \\
  $\BigOmega{f}$
    & $g = \BigOmega{f} \iff f = \BigO{g}$. \\
  $\LittleOmega{f}$
    & $g = \LittleOmega{f} \iff f = \LittleO{g}$. \\
\end{tabularx}
\endgroup

\begingroup
\renewcommand*{\arraystretch}{1.5}
\begin{tabularx}{\linewidth}{llX}
  Distribution & Domain & Density function $f(x)$ \\
  \midrule
  \rule{0pt}{5ex}%
  $\DNorm{\bm{\mu}, \bm{\Sigma}}$
    & $\mathbf{R}^D$
    & $\displaystyle \frac{1}{\sqrt{2 \pi \Abs{\bm{\Sigma}}}}
        \exp \left[ -\frac{1}{2} (\bm{x} - \bm{\mu})^\Tr \bm{\Sigma}^{-1} (\bm{x} - \bm{\mu}) \right]$ \\
  \rule{0pt}{5ex}%
  $\DUnif{a, b}$
    & $[a, b]$
    & $\displaystyle \frac{1}{b - a}$ \\
  \rule{0pt}{5ex}%
  $\DBeta{\alpha, \beta}$
    & $[0, 1]$
    & $\displaystyle \frac{1}{B(\alpha, \beta)} x^{\alpha - 1} (1-x)^{\beta - 1}$ \\
  $\DExp{\lambda}$
    & $\mathbf{R}_{>0}$
    & $\displaystyle \lambda \exp(- \lambda x)$ \\
  \rule{0pt}{5ex}%
  $\DGamma{\alpha, \beta}$
    & $\mathbf{R}_{>0}$
    & $\displaystyle \frac{\beta^\alpha}{\Gamma(\alpha)} x^{\alpha-1} \exp(- \beta x)$ \\
  \rule{0pt}{5ex}%
  $\DGumbel{\mu, \beta}$
    & $\mathbf{R}$
    & $\displaystyle \frac{1}{\beta} \exp \{ - [z + \exp(-z)]\}$, where $\displaystyle z = \frac{x - \mu}{\beta}$ \\
\end{tabularx}
\endgroup
