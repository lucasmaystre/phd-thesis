\cleardoublepage
\chapter*{Mathematical Notation}
%\markboth{Abstract}{Abstract}  % TODO: Do we need this if the abstract takes several pages?
\addcontentsline{toc}{chapter}{Mathematical Notation} % adds an entry to the table of contents

Boldface lowercase, $\bm{x} = [x_i]$ to denote a column vector.
Boldface uppercase, $\bm{M} = [m_ij]$ for matrices.
Standard sets of numbers: $\mathbf{R}$ the real numbers, $\mathbf{R}_{>0}$ the positive reals, $\mathbf{N}$ the natural numbers.
Calligraphic letters such as $\mathcal{A}$, $\mathcal{S}$ or $\mathcal{D}$ are used for other sets.
$\Prob{\mathcal{A}}$ the probability of event $\mathcal{A}$.
The set of the $N$ first consecutive  natural numbers is written $[N] \doteq \{ 1, \ldots, N \}$
$\Exp[q]{x}$ the expectation of the random variable $x$ under the distribution $q$, similarly $\Var[q]{x}$ the variance of $x$ and $\Cov[q]{x, y}$ the covariance of $x$ and $y$.
In many cases, the distribution $q$ is omitted and is clear from the context.

We denote by $i \succ j$ the event ``$i$ wins over $j$'' in a pairwise comparison.
Similarly, we denote by $i \succeq \mathcal{A}$ the event ``$i$ is chosen among alternatives $\mathcal{A}$''.
