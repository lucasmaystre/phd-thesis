\section{Related Work}
\label{pk:sec:relwork}

More than two decades after Zermelo's seminal paper \citep{zermelo1928berechnung}, his model for paired comparisons was rediscovered and popularized by \citet{bradley1952rank}.
Nowadays, the model is usually referred to as the Bradley--Terry model.
In the context of skill-based game modeling, the same model (associated to a simple online stochastic gradient update rule) is also known as the Elo rating system \citep{elo1978rating}.
It is used by FIDE to rank chess players\footnote{See: \url{https://ratings.fide.com/}.} and by FIFA to rank women national football teams\footnote{See: \url{http://www.fifa.com/fifa-world-ranking/procedure/women.html}.}, among others.

The model and related inference algorithms have been extended in various ways; one direction that is of particular interest is the handling of uncertainty of the estimated skill parameters.
\citet{glickman1999parameter} proposes an extension that simultaneously updates ratings and associated uncertainty values after each observation.
\citet{herbrich2006trueskill} propose TrueSkill, a comprehensive Bayesian framework for estimating player skill in various types of games.
The models and methods described in this paper are fundamentally similar to TrueSkill, as will be discussed in Section~\ref{pk:sec:methods}.
Finally, in the context of learning users' preferences from pairwise comparisons, \citet{chu2005preference} present a Gaussian process approach that is comparable to our work.
