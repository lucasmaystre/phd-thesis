\chapter{Predicting Football Matches}
\label{ch:playerkern}

In this chapter\footnote{%
This chapter is based on \citet*{maystre2016player}.},
we shift our attention from human choices to sports outcomes.
In particular, we draw attention to a connection between skill-based models of game outcomes (built on the Bradley--Terry model) and Gaussian-process classification models.
The Gaussian-process perspective enables
\begin{enuminline}
\item a principled way of dealing with uncertainty and
\item rich models, specified through kernel functions.
\end{enuminline}
Using this connection, we tackle the problem of predicting outcomes of football matches between national teams.
We develop a \emph{player kernel} that relates any two football matches through the players lined up on the field.
This makes it possible to share knowledge gained from observing matches between clubs (available in large quantities) and matches between national teams (available only in limited quantities).
We evaluate our approach on the Euro 2008, 2012 and 2016 final tournaments.


%%%%%%%%%%%%%%%%%%%%%%%%%%%%%%%%%%%%%%%%%%%%%%%%%%%%%%%%%%%%%%%%%%%%%%%%%
\section{Introduction}  %%%%%%%%%%%%%%%%%%%%%%%%%%%%%%%%%%%%%%%%%%%%%%%%%
\label{sec:intro}

The problem of recovering a ranking over $n$ items from noisy outcomes of pairwise comparisons has attracted, in the last century, much research interest, driven by applications in sports \citep{elo1978rating}, social sciences \citep{thurstone1927law, salganik2015wiki} and---more recently---recommender systems \citep{houlsby2012collaborative}.
Whereas pairwise comparison models and related inference algorithms have been extensively studied, the issue of \emph{which pairwise comparisons to sample}, also known as active learning, has received significantly less attention.
To understand the potential benefits of adaptively selecting samples, consider the case where comparison outcomes are noiseless, i.e., consistent with a linear order on a set of $n$ items.
If pairs of items are selected at random, it is necessary to collect \BigOmega{n^2} comparisons to recover the ranking \citep{alon1994linear}.
In contrast, by using an efficient sorting algorithm, \BigO{n \log n} adaptively chosen comparisons are sufficient.
In this work, we demonstrate that sorting algorithms can also be helpful in the \emph{noisy} setting, where some comparison outcomes are inconsistent with the ranking: despite errors, sorting algorithms tend to select informative samples.
We focus on the Bradley--Terry (BT) model, a widely-used probabilistic model of comparison outcomes.
In this model, each item is associated with a parameter on the real line, and the probability of observing an incorrect outcome decreases as the distance between the items' parameters increases.

First, we study the output of a single execution of Quicksort when comparison outcomes are generated from a BT model, under the assumption that the distance between adjacent parameters is (stochastically) uniform across the ranking.
We measure the quality of a ranking estimate by its displacement with respect to the ground truth, i.e., the sum of rank differences.
We show that Quicksort's output is a good approximation to the ground-truth ranking: no method comparing every pair of items at most once can do better (up to constant factors).
Furthermore, we show that by aggregating \BigO{\log^5 n} independent runs of Quicksort, it is possible to recover the exact rank for all but a vanishing fraction of the items.
These theoretical results suggest that adaptive sampling is able to bring a substantial acceleration to the learning process.

Second, we propose a practical active-learning (AL) strategy that consists of repeatedly sorting the items.
We evaluate our sorting-based method on three datasets and compare it to existing AL methods.
We observe that \emph{all} the strategies that we consider lead to better ranking estimates noticeably faster than random sampling.
However, most strategies are challenging to operate and computationally expensive, thus hindering wider adoption \citep{schein2007active}.
In this regard, sorting-based AL stands out, as
\begin{enuminline}
\item it is computationally-speaking as inexpensive as random sampling, 
\item it is trivial to implement, and
\item it requires no tuning of hyperparameters.
\end{enuminline}

\subsection{Preliminaries and Notation}

We consider $n$ items that are represented by consecutive integers $[n] = \{1, \ldots, n\}$.
Without loss of generality, we assume that the items are ranked by increasing preference\footnote{
This convention greatly simplifies the notation throughout the paper, but differs from that used in most of the preference learning literature.
In our paper, the item with rank $1$ is the \emph{worst}.}, i.e., $i < j$ means that $j$ is (in expectation) preferred to $i$.
When $j$ is preferred to $i$ as a result of a pairwise comparison, we denote the observation by $i \prec j$.
If $i < j$, we say that $i \prec j$ is a \emph{consistent} outcome and $j \prec i$ an \emph{inconsistent} (incorrect) outcome.
In most of the paper, pairwise comparison outcomes follow a Bradley--Terry model with parameters $\bm{\theta} = \begin{bmatrix} \theta_1 & \cdots & \theta_n \end{bmatrix} \in \Set{R}^n$, denoted $\BT(\bm{\theta})$.
The parameters $\theta_1 < \cdots < \theta_n$ represent the utilities of items $1, \ldots, n$, and the probability of observing the outcome $i \prec j$ is
\begin{align*}
p(i \prec j \mid \bm{\theta}) = \frac{1}{1 + \exp[-(\theta_j - \theta_i)]}.
\end{align*}
The probability of observing an inconsistent comparison decreases with the distance between the items.
This captures the intuitive notion that some pairs of items are easy to compare and some are more difficult \citep{zermelo1928berechnung, bradley1952rank}.

A ranking $\sigma$ is a function that maps an item to its rank, i.e., $\sigma(i) =$ rank of item $i$.
The (ground-truth) identity ranking is denoted by \id, i.e. $\id(i) = i$.
To measure the quality of a ranking $\sigma$ with respect to the ground-truth, we consider the \emph{displacement}
\begin{align*}
\Disp{\sigma} = \sum_{i=1}^n | \sigma(i) - i |,
\end{align*}
also known as Spearman's footrule distance.
Another metric widely used in practice is the Kendall--Tau distance, defined as
$K(\sigma) = \sum_{i < j} \Indic{\sigma(i) > \sigma(j)}$.
Both metrics are equivalent up to a factor of two\footnote{$\Disp{\sigma} / 2 \le K(\sigma) \le \Disp{\sigma}$ \citep{diaconis1977spearman}.}, such that bounds on \Disp{\sigma} also hold for $K(\sigma)$ up to constant factors.

Finally, we say that an event $A$ holds \emph{with high probability} if $\Prob{A} \to 1$ as $n \to \infty$.
For a random variable $X$ and a sequence of numbers $a_n$, we say that $X = \BigO{a_n}$ with high probability if $\Prob{\Abs{X} \le c a_n} \to 1$ as $n \to \infty$ for some constant $c$ that does not depend on $n$.

\paragraph{Outline of the paper.}
We begin by briefly reviewing related literature in Section~\ref{sec:relwork}.
Next, in Section~\ref{sec:theory}, we study the displacement of Quicksort's output under noisy comparisons.
In Section~\ref{sec:experiments}, we empirically evaluate several AL strategies on three datasets.
Finally, we conclude in Section~\ref{sec:conclusion}.

\section{Related Work}
\label{pk:sec:relwork}

Zermelo's \citeyear{zermelo1928berechnung} paper (discussed in Section~\ref{in:sec:btmodel}) presented the first statistical model of chess game outcomes.
His model, associated with a simple online stochastic gradient update rule, is known as the Elo rating system \citep{elo1978rating}.
This rating system is currently used by the World Chess Federation (FIDE) to rank chess players\footnote{See: \url{https://ratings.fide.com/}.} and by the International Federation of Football Association (FIFA) to rank women's national football teams\footnote{See: \url{http://www.fifa.com/fifa-world-ranking/procedure/women.html}.}, among others.

The model and related inference algorithms have been extended in various ways, e.g., by considering other types of outcomes \citep{rao1967ties, maher1982modelling} or by permitting parameters to evolve over time \citep{glickman1993paired, fahrmeir1994dynamic, cattelan2013dynamic}.
One direction that is of particular interest in this chapter is the handling of the uncertainty of the estimated skill parameters.
\citet{glickman1999parameter} proposes an extension that simultaneously updates ratings and associated uncertainty values, after each observation, by using a simple closed-form update.
\citet{herbrich2006trueskill} propose TrueSkill, a comprehensive Bayesian framework for estimating player skills in various types of games based on the expectation-propagation algorithm.
The models and methods described in this chapter are similar to TrueSkill, as will be discussed in Section~\ref{pk:sec:methods}.
In the context of learning users' preferences from pairwise comparisons, \citet{chu2005preference} were the first to link the Bayesian treatment of pairwise comparisons models to Gaussian-process classification.

\section{Methods}
\label{pk:sec:methods}

In this section, we first show how the Bradley--Terry model of pairwise comparisons (the modern name of Zermelo's model), can be expressed in the Gaussian-process framework.
The Gaussian-process viewpoint shifts the focus from \emph{items} (or, in our case, contestants) to \emph{games}: the statistical relationship between outcomes of several games is given by a covariance function.
Second, we present the \emph{player kernel}, a covariance function that relates football matches through lineups.


\subsection{Gaussian-Process Classification Viewpoint}

Suppose that we observe outcomes of comparisons between two items (e.g., two players or two teams) in a universe of items denoted $1, \ldots, N$.
We begin by restricting ourselves to binary outcomes, i.e., we assume that one of the two items necessarily wins.
The Bradley-Terry model postulates that each item $i$ can be represented by a parameter $\theta_i \in \mathbf{R}$, indicative of its relative strength against an opponent.
Given these parameters, the probability of observing the outcome $i \succ j$ is given by
\begin{align}
\label{pk:eq:logistic}
\Prob{i \succ j} = \frac{1}{1 + \exp[-(\theta_i - \theta_j)]} = \frac{1}{1 + \exp(- \bm{\theta}^\Tr \bm{x})},
\end{align}
where $\bm{\theta} = [\theta_i]$ and $\bm{x} \in \mathbf{R}^N$ is such that $x_i = 1$, $x_j = -1$ and $x_k = 0$ for $k \ne i, j$.
As such, the pairwise comparison model can be seen as a special case of logistic regression \citep[Chapter 4]{bishop2006pattern}, where the feature vector simply indicates the winning and losing items.
Furthermore, logistic regression is itself a special case of Gaussian-process classification \cite[Chapter 3]{rasmussen2006gaussian}.

\begin{definition}[Gaussian process]
A \emph{Gaussian process}
\begin{align*}
f(\bm{x}) \sim \mathrm{GP}[m(\bm{x}), k(\bm{x}, \bm{x}')]
\end{align*}
is a stochastic process defined by a mean function $m(\bm{x}) \doteq \Exp{f(\bm{x})}$ and a positive semi-definite covariance (or kernel) function $k(\bm{x}, \bm{x}') \doteq \Cov{f(\bm{x})}{f(\bm{x}')}$.
Given any finite collection of points $\bm{x}_1, \ldots, \bm{x}_M$, the Gaussian process sampled at these points has a multivariate Gaussian distribution
\begin{align*}
\begin{bmatrix}
f(\bm{x}_1) & \cdots & f(\bm{x}_M)
\end{bmatrix} \sim \DNorm{\bm{m}, \bm{K}},
\end{align*}
where $m_u = m(\bm{x}_u)$ and $k_{uv} = k(\bm{x}_u, \bm{x}_v)$.
\end{definition}

It is not hard to show that if $\bm{\theta} \sim \DNorm{\bm{0}, \sigma^2 \bm{I}}$, then $f(\bm{x}) = \bm{\theta}^\Tr \bm{x}$ is a Gaussian process with $m(\bm{x}) = 0$ and $k(\bm{x}, \bm{x}') = \sigma^2 \bm{x}^\top \bm{x}'$.
This enables us to interpret \eqref{pk:eq:logistic} as the likelihood of a Gaussian-process classification model with the logit link function.

The Gaussian-process viewpoint shifts the focus from the parametric representation of the function $f(\bm{x})$ (in the case of \eqref{pk:eq:logistic}, a linear function of items strengths) to the covariance between two function evaluations, as defined by the kernel function $k(\bm{x}, \bm{x}')$.
Intuitively (and informally), the model can simply be specified by stating how similar any two match outcomes are expected to be.
Furthermore, the Gaussian-process viewpoint also makes it possible to take advantage of the vast amount of literature and software related to accurate, efficient, and scalable inference.


\paragraph{Handling Draws}
\citet{rao1967ties} propose an extension of the pairwise comparison model for ternary (win, draw, loss) outcomes.
In this extension, the two different types of outcomes have probabilities
\begin{align*}
\Prob{i \succ j}  &= \frac{1}{1 + \exp[f(\bm{x}) - \alpha]} \\
\Prob{i \equiv j} &= (e^{2 \alpha} - 1) \Prob{i \succ j} \Prob{j \succ i},
\end{align*}
where $\alpha > 0$ is an additional hyperparameter controlling the frequency of draws (see also Section~\ref{fi:sec:ties}).
Because a draw can be written as the product of a win and a loss, model inference can still be performed using only a \emph{binary} Gaussian-process classification model, with the changes needed to the link function being minimal.


\subsection{The Player Kernel}

We now consider an application to football and propose a method to quantify how similar two match outcomes are expected to be.
Let $1, \dots, P$ denote all distinct players appearing in a dataset of matches.
We define a team's \emph{lineup} as the set consisting of the \num{11} players starting the match.
For a given match, let $\mathcal{W}$ and $\mathcal{L}$ be the lineups of the winning and losing teams, respectively.
Define $\bm{z} \in \mathbf{R}^P$ such that $z_p = 1$ if $p \in \mathcal{W}$, $z_p = -1$ if $p \in \mathcal{L}$ and $z_p = 0$ otherwise.
We then define the player kernel as
\begin{align*}
k(\bm{z}, \bm{z}') = \sigma^2 \bm{z}^\top \bm{z}'.
\end{align*}
Intuitively, the function is positive if the same players are lined up in both matches, and the same players win (respectively, lose).
The function is negative when players win one match, but lose the other.
Finally, the function is zero, e.g., when the lineups are completely disjoint.

This kernel implicitly projects every match into the space of players, and defines a notion of similarity in this space.
In the case of national teams qualified to Euro final tournaments, we find that this approach is very useful: a significant part of national teams' players take part in one of the main European leagues and play with or against each other.
International club competitions (such as the UEFA Champions League) further contribute to the ``connectivity'' among players.
Figure~\ref{pk:fig:kernel} illustrates the similarity of matches across different competitions in 2011--2012.


\begin{figure}
  \centering
  \includegraphics{pk-kernelmatrix}
  \caption{Heatmap of the magnitude of the kernel matrix for \num{3184} matches played over the year preceding Euro 2012.
White indicates zero correlation, black indicates non-zero correlation.
Matches between national teams exhibit non-zero covariance with matches of all other competitions.
}
  \label{pk:fig:kernel}
\end{figure}

It is interesting to note that the player kernel corresponds to a linear model over the players.
That is, it is equivalent to assuming that there is one independent skill parameter per player, and that the strength of a team is the sum of its players' skills.
Such a model contains a massive number of parameters (possibly much more than the number of observations), and there is little hope for a reliable estimation of every parameter.
In fact, in Section~\ref{pk:sec:evaluation} we observe that the model is ``weakly'' parametric: the number of distinct players usually grows with the number of matches observed.
The kernel-based viewpoint that we take emphasizes the fact that estimating these parameters explicitly is \emph{not} necessary.

\paragraph{Relation to TrueSkill}
Our Gaussian-process model coupled with the player kernel is very similar to TrueSkill \citep{herbrich2006trueskill}.
The most important difference is that we take advantage of the dual representation and operate in the space of matches, instead of in the space of players.
Beyond the conceptual reasons outlined above, the model makes inference less computationally intensive for the datasets that we consider.

\section{Experimental Evaluation}
\label{pk:sec:evaluation}

In this section, we evaluate our predictive model on the matches of the Euro 2008, 2012 and 2016 final tournaments and compare it to several baselines.

We collect a dataset of matches from
\begin{enuminline}
\item official and friendly competitions involving national teams, and
\item the most prestigious European club competitions,
\end{enuminline}
starting from July 1\textsuperscript{st}, 2006.
The list of competitions is displayed in Table~\ref{pk:tab:competitions}.
There are approximately $15 \times$ more matches between clubs than there are matches between national teams in our dataset.
With respect to the model outlined in Section~\ref{pk:sec:methods}, our final predictive model processes one additional feature that encodes which team played at home (this feature is null for matches played on neutral ground).
We train the model using a dataset $\mathcal{D}$ consisting of all $M$ matches that were played prior to the start of the competition on which we test.
When computing the kernel matrix (whether on training or on test data) we use the starting lineups, usually announced shortly before the start of the match.
It is interesting to note that the number of distinct players $P$ appearing in the dataset exceeds the number of training instances in each case (the values of $M$ and $P$ are shown in Table~\ref{pk:tab:eval}).

Starting from a Gaussian prior distribution over the $M$ matches $\bm{f} = [f_1 \ \cdots \ f_M]^\Tr \sim \DNorm{\bm{f} \mid \bm{m}, \bm{K}}$, we seek to find the posterior distribution
\begin{align*}
p(\bm{f} \mid \mathcal{D}) \propto \DNorm{\bm{f} \mid \bm{m}, \bm{K}} \prod_{m = 1}^M \frac{1}{1 + \exp(- f_m)}.
\end{align*}
This distribution is intractable, and we use the expectation propagation algorithm\footnote{%
We use the GPy Python library (see: \url{https://sheffieldml.github.io/GPy/}) to fit the model; inference takes a minute for the 2008 test set (17 minutes for 2016).}
to approximate it by a multivariate normal distribution \citep{minka2001family}.
Once the posterior is computed, we can use it to generate predictions for new matches \citep{rasmussen2006gaussian}.
These predictions come in the form of probability distributions $[p^{\text{W}}, p^{\text{D}}, p^{\text{L}}]$ over the three outcomes (win, draw, loss).

\begin{table}
  \caption{
List of competitions included in the dataset, spanning matches from 2006 to 2016.
The majority of matches are played in competitions between clubs.}
  \label{pk:tab:competitions}
  \centering
  \begin{tabular}{llc}
    \toprule
    Competition           & Country       & Involves clubs      \\
    \midrule
    Bundesliga            & Germany       & $\bullet$     \\
    Confederations Cup    & International & \\
    EC Qualification      & International & \\
    European Championship & International & \\
    Friendlies            & International & \\
    Ligue 1               & France        & $\bullet$     \\
    Premier League        & England       & $\bullet$     \\
    La Liga               & Spain         & $\bullet$     \\
    Serie A               & Italy         & $\bullet$     \\
    UEFA Champions League & International & $\bullet$     \\
    UEFA Europa League    & International & $\bullet$     \\
    World Cup             & International & \\
    \bottomrule
  \end{tabular}
\end{table}

We compare our predictive distributions against three baselines.
First, we consider a simple Rao-Kupper model based on national team ratings obtained from a popular Web site\footnote{See: \url{http://www.eloratings.net/}.}.
This model is similar to ours, but
\begin{enuminline}
\item it does not relate matches through players, and thus does not consider club outcomes, and
\item as ratings are fixed values, it does not consider uncertainty in the ratings.
\end{enuminline}
Second, we consider average probabilities derived from the odds given by three large betting companies.
Third, we consider a random baseline which always outputs $[1/3, 1/3, 1/3]$.
The predictive distributions are evaluated using the average logarithmic loss over $T$ test instances
\begin{align*}
- \frac{1}{T} \sum_{i=1}^{T} \left[
\Indic{y_i = \text{W}} \log p^{\text{W}}_i
+ \Indic{y_i = \text{D}} \log p^{\text{D}}_i
+ \Indic{y_i = \text{L}} \log p^{\text{L}}_i
\right].
\end{align*}
The logarithmic loss penalizes more strongly predictions that are both confident and incorrect.
Table~\ref{pk:tab:eval} summarizes the results.


\begin{table}
  \caption{
  Average logarithmic loss of our predictive model (PlayerKern), a model based on national team ratings (Elo), betting odds (Odds) and a random baseline (Random) on the final tournaments of three European championships.
  $M$ is the number of training instances, $P$ the number of distinct players and $T$ the number of test instances.}
  \label{pk:tab:eval}
  \centering
  \begin{tabular}{l rr rrrrr}
    \toprule
    Competition & $M$         & $P$         & $T$      & PlayerKern           & Elo                  & Odds                 &  Random \\
    \midrule
    Euro 2008   & \num{4390}  & \num{7875}  & \num{31} & \num{0.969}          & \textbf{\num{0.910}} & \num{0.979}          & \num{1.099} \\
    Euro 2012   & \num{15594} & \num{21735} & \num{31} & \textbf{\num{0.939}} & \num{1.003}          & \num{0.953}          & \num{1.099} \\
    Euro 2016   & \num{24887} & \num{33157} & \num{51} & \num{1.067}          & \num{1.102}          & \textbf{\num{1.020}} & \num{1.099} \\
    \bottomrule
  \end{tabular}
\end{table}

Our predictive model performs well in 2008 and 2012, but slightly less so in 2016.
It is noteworthy that the 2016 final tournament has been generally less predictable than earlier editions.
The case of the Elo baseline is interesting, as its accuracy varies wildly.
Reasons for this might include the noise due to the online gradient updates, and the lack of proper uncertainty quantification in the ratings.
Our method, in contrast, seems to produce more conservative predictions, but manages to achieve a more consistent performance.

%%%%%%%%%%%%%%%%%%%%%%%%%%%%%%%%
\section{Summary}
\label{fi:sec:summary}

In this chapter, we develop a stationary-distribution perspective on the maximum-likelihood estimate of Luce's choice model.
This perspective explains and unifies several recent spectral algorithms from an ML inference point of view.
We present our own spectral algorithm that works on a wider range of data, and show that the resulting estimate significantly outperforms previous approaches in terms of accuracy.
We also show that this simple algorithm, with a straighforward adaptation, can produce a sequence of estimates that converge to the ML estimate.
On real-world datasets, our ML algorithm is always faster than the state of the art, at times by up to two orders of magnitude.

Beyond statistical and computational performance, we believe that a key strength of our algorithms is that they are simple to implement.
As an example, our implementation of LSR fits in ten lines of Python code.
The most complex operation---finding a stationary distribution---can be readily offloaded to commonly available and highly optimized linear-algebra primitives.
As such, we believe that our contribution is useful for practitioners.

