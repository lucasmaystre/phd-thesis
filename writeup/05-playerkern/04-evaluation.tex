\section{Experimental Evaluation}
\label{pk:sec:evaluation}

In this section, we evaluate our predictive model on the matches of the Euro 2008, 2012 and 2016 final tournaments and compare it to several baselines.

We collect a dataset of matches from
\begin{enuminline}
\item official and friendly competitions involving national teams, and
\item the most prestigious European club competitions,
\end{enuminline}
starting from July 1\textsuperscript{st}, 2006.
The list of competitions is displayed in Table~\ref{pk:tab:competitions}.
There are approximately $15 \times$ more matches between clubs than there are matches between national teams in our dataset.
With respect to the model outlined in Section~\ref{pk:sec:methods}, our final predictive model processes one additional feature that encodes which team played at home (this feature is null for matches played on neutral ground).
We train the model using a dataset $\mathcal{D}$ consisting of all $M$ matches that were played prior to the start of the competition on which we test.
When computing the kernel matrix (whether on training or on test data) we use the starting lineups, usually announced shortly before the start of the match.
It is interesting to note that the number of distinct players $P$ appearing in the dataset exceeds the number of training instances in each case (the values of $M$ and $P$ are shown in Table~\ref{pk:tab:eval}).

Starting from a Gaussian prior distribution over the $M$ matches $\bm{f} = [f_1 \ \cdots \ f_M]^\Tr \sim \DNorm{\bm{f} \mid \bm{m}, \bm{K}}$, we seek to find the posterior distribution
\begin{align*}
p(\bm{f} \mid \mathcal{D}) \propto \DNorm{\bm{f} \mid \bm{m}, \bm{K}} \prod_{m = 1}^M \frac{1}{1 + \exp(- f_m)}.
\end{align*}
This distribution is intractable, and we use the expectation propagation algorithm\footnote{%
We use the GPy Python library (see: \url{https://sheffieldml.github.io/GPy/}) to fit the model; inference takes a minute for the 2008 test set (17 minutes for 2016).}
to approximate it by a multivariate normal distribution \citep{minka2001family}.
Once the posterior is computed, we can use it to generate predictions for new matches \citep{rasmussen2006gaussian}.
These predictions come in the form of probability distributions $[p^{\text{W}}, p^{\text{D}}, p^{\text{L}}]$ over the three outcomes (win, draw, loss).

\begin{table}
  \caption{
List of competitions included in the dataset, spanning matches from 2006 to 2016.
The majority of matches are played in competitions between clubs.}
  \label{pk:tab:competitions}
  \centering
  \begin{tabular}{llc}
    \toprule
    Competition           & Country       & Involves clubs      \\
    \midrule
    Bundesliga            & Germany       & $\bullet$     \\
    Confederations Cup    & International & \\
    EC Qualification      & International & \\
    European Championship & International & \\
    Friendlies            & International & \\
    Ligue 1               & France        & $\bullet$     \\
    Premier League        & England       & $\bullet$     \\
    La Liga               & Spain         & $\bullet$     \\
    Serie A               & Italy         & $\bullet$     \\
    UEFA Champions League & International & $\bullet$     \\
    UEFA Europa League    & International & $\bullet$     \\
    World Cup             & International & \\
    \bottomrule
  \end{tabular}
\end{table}

We compare our predictive distributions against three baselines.
First, we consider a simple Rao-Kupper model based on national team ratings obtained from a popular Web site\footnote{See: \url{http://www.eloratings.net/}.}.
This model is similar to ours, but
\begin{enuminline}
\item it does not relate matches through players, and thus does not consider club outcomes, and
\item as ratings are fixed values, it does not consider uncertainty in the ratings.
\end{enuminline}
Second, we consider average probabilities derived from the odds given by three large betting companies.
Third, we consider a random baseline which always outputs $[1/3, 1/3, 1/3]$.
The predictive distributions are evaluated using the average logarithmic loss over $T$ test instances
\begin{align*}
- \frac{1}{T} \sum_{i=1}^{T} \left[
\Indic{y_i = \text{W}} \log p^{\text{W}}_i
+ \Indic{y_i = \text{D}} \log p^{\text{D}}_i
+ \Indic{y_i = \text{L}} \log p^{\text{L}}_i
\right].
\end{align*}
The logarithmic loss penalizes more strongly predictions that are both confident and incorrect.
Table~\ref{pk:tab:eval} summarizes the results.


\begin{table}
  \caption{
  Average logarithmic loss of our predictive model (PlayerKern), a model based on national team ratings (Elo), betting odds (Odds) and a random baseline (Random) on the final tournaments of three European championships.
  $M$ is the number of training instances, $P$ the number of distinct players and $T$ the number of test instances.}
  \label{pk:tab:eval}
  \centering
  \begin{tabular}{l rr rrrrr}
    \toprule
    Competition & $M$         & $P$         & $T$      & PlayerKern           & Elo                  & Odds                 &  Random \\
    \midrule
    Euro 2008   & \num{4390}  & \num{7875}  & \num{31} & \num{0.969}          & \textbf{\num{0.910}} & \num{0.979}          & \num{1.099} \\
    Euro 2012   & \num{15594} & \num{21735} & \num{31} & \textbf{\num{0.939}} & \num{1.003}          & \num{0.953}          & \num{1.099} \\
    Euro 2016   & \num{24887} & \num{33157} & \num{51} & \num{1.067}          & \num{1.102}          & \textbf{\num{1.020}} & \num{1.099} \\
    \bottomrule
  \end{tabular}
\end{table}

Our predictive model performs well in 2008 and 2012, but slightly less so in 2016.
It is noteworthy that the 2016 final tournament has been generally less predictable than earlier editions.
The case of the Elo baseline is interesting, as its accuracy varies wildly.
Reasons for this might include the noise due to the online gradient updates, and the lack of proper uncertainty quantification in the ratings.
Our method, in contrast, seems to produce more conservative predictions, but manages to achieve a more consistent performance.
