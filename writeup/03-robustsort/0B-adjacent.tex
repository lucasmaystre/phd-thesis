%%%%%%%%%%%%%%%%%%%%%%%%%%%%%%%%%%%%%%%%%%%%%%%%%%%%%%%%%%%%%%%%%%%%%%%%%
\section{Discriminating the Closest Items}  %%%%%%%%%%%%%%%%%%%%%%%%%%%%%
\label{app:adjacent}


The distance between the two closest items is $d_{\min} = \min_i \Abs{\theta_{i+i} - \theta{i}} = \min_i x_i$, i.e., the minimum of $n-1$ independent exponential random variables of rate $\lambda$.
Therefore, $d_{\min} \sim \DExp{(n-1)\lambda}$, and for $n \ge 2$ with probability at least $1 - e^{-1/2} \approx 0.39$ we have $d_{\min} \le (\lambda n)^{-1}$.
Suppose that we compare the two closest items $m$ times, and let $z_i$ be the indicator random variable for the event ``the outcome of the $i$-th comparison is incorrect''.
Assuming that $d_{\min} \le (\lambda n)^{-1}$ and that $\lambda n \ge 1/2$,
\begin{align*}
\Prob{z_i = 0}
    \le \frac{1}{1 + \exp[-1 / (\lambda n)]} \le \frac{1}{2 - 1/(\lambda n)}
    = \frac{1}{2} \cdot \left( 1 + \frac{1}{2\lambda n - 1} \right)
    \le \frac{1}{2} \exp \left[ \frac{1}{2\lambda n - 1} \right],
\end{align*}
where we used the inequality $e^{x} \ge 1 + x$ twice.
Given the $m$ comparison outcomes, we use a majority vote to decide the relative order of the two items.
The probability of making the correct decision is
%Based on these $m$ outcomes, we use a majority vote to decide the respective order of the items.
\begin{align*}
\Prob{\sum_{i = 1}^m z_i \le m/2}
    &\le \sum_{k = 1}^{m/2} \binom{m}{k} \Prob{z_i = 0}^m
     \le \exp \left[ \frac{m}{2\lambda n - 1} \right] \cdot 2^{-m} \sum_{k = 1}^{m/2} \binom{m}{k} \\
    &= \frac{1}{2} \exp \left[ \frac{m}{2\lambda n - 1} \right].
\end{align*}
Therefore, if $m = \LittleO{\lambda n}$ the probability of making a mistake is bounded from below by a positive constant.
