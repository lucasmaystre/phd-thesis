%%%%%%%%%%%%%%%%%%%%%%%%%%%%%%%%%%%%%%%%%%%%%%%%%%%%%%%%%%%%%%%%%%%%%%%%%
\section{ChoiceRank Algorithm}  %%%%%%%%%%%%%%%%%%%%%%%%%%%%%%%%%%%%%%%%%
\label{cr:app:algorithm}

In this section, we start by generalizing the ChoiceRank algorithm to the weighted network choice model.
We then prove the convergence of this generalized algorithm.
Finally, we show how the same algorithm can be obtained from an EM viewpoint by introducing suitable latent variables.

%%%%%%%%%%%%%%%%%%%%%%%%%%%%%%%%%%%%%%%%%%%%%%%%%%%%%%%%%%%%%%%%%%%%%%%%%
\subsection{Algorithm for the Generalized Model}

Using the same linear upper-bound on the logarithm as in Section~\ref{cr:sec:algorithm} of the main text, we can lower-bound the log-posterior~\eqref{cr:eq:wlogpost} in the weighted model by
\begin{align}
\label{cr:eq:wminorizing}
\begin{aligned}
f^{(t)}(\bm{\gamma}) = \kappa_2 + \sum_{i = 1}^N \bigg[
    & (c^-_i + \alpha - 1) \log \gamma_i - \beta \gamma_i \\
    &- c^+_i \bigg( \log\!\sum_{k \in \mathcal{N}^+_i}\!w_{ik} \gamma^{(t)}_k
                   +\frac{\sum_{k \in \mathcal{N}^+_i}\!w_{ik} \gamma_k}{\sum_{k \in \mathcal{N}^+_i}\!w_{ik} \gamma^{(t)}_k} -1 \bigg) \bigg],
\end{aligned}
\end{align}
with equality if and only if $\bm{\gamma} = \bm{\gamma}^{(t)}$.
Starting with an arbitrary $\bm{\gamma}^{(0)} \in \mathbf{R}^N_{>0}$, we repeatedly maximize the lower-bound $f^{(t)}$.
This surrogate optimization problem has a closed form solution, obtained by setting $\nabla f^{(t)}$ to $0$:
\begin{align}
\label{cr:eq:wmmiter}
\gamma_i^{(t + 1)} = \frac{c^-_i + \alpha - 1}{\sum_{j \in \mathcal{N}^-_i} w_{ji} \mu_j^{(t)} + \beta},
    \quad \text{where }
    \mu_j^{(t)} = \frac{c^+_j}{\sum_{k \in \mathcal{N}^+_j} w_{jk} \gamma_k^{(t)}}.
\end{align}
The iterates provably converge to the maximizer of~\eqref{cr:eq:wlogpost}, as shown by the following theorem.

\begin{theorem}
\label{cr:thm:wmmconv}
Let $\bm{\gamma}^\star$ be the unique maximum a-posteriori estimate.
Then for any initial $\bm{\gamma}^{(0)} \in \mathbf{R}^N_{> 0}$ the sequence of iterates defined by~\eqref{cr:eq:wmmiter} converges to $\bm{\gamma}^\star$.
\end{theorem}

The proof follows that of \citeauthor{hunter2004mm}'s Theorem~$1$ \citeyearpar{hunter2004mm}.

\begin{proof}
Let $M: \mathbf{R}^N_{>0} \to \mathbf{R}^N_{>0}$ be the (continuous) map implicitly defined by one iteration of the algorithm.
For conciseness, let $g(\bm{\gamma}) \doteq \log p(\bm{\gamma} \mid \mathcal{D})$.
As $g$ has a unique maximizer and is concave using the reparametrization $\gamma_i = e^{\theta_i}$, it follows that $g$ has a single stationary point.
First, observe that the minorization-maximization property guarantees that $g \left[ M(\bm{\gamma}) \right] \ge g(\bm{\gamma})$.
Combined with the strict concavity of $g$, this ensures that $\lim_{t \to \infty} g(\bm{\gamma}^{(t)})$ exists and is unique for any $\bm{\gamma}^{(0)}$.
Second, $g \left[ M(\bm{\gamma}) \right] = g(\bm{\gamma})$ if and only if $\bm{\gamma}$ is a stationary point of $g$, because the minorizing function is tangent to $g$ at the current iterate.
It follows that $\lim_{t \to \infty} \bm{\gamma}^{(t)} = \bm{\gamma}^{\star}$.
\end{proof}

Theorem~\ref{cr:thm:mmconv} of the main text follows directly by setting $w_{ij} \equiv 1$.
For completeness, the edge-streaming implementation adapted to the weighted model is given in Algorithm~\ref{cr:alg:wchoicerank}.
The only changes with respect to Algorithm~\ref{cr:alg:choicerank} (presented in the main text) are in lines~\ref{cr:line:msg1} and~\ref{cr:line:msg2}:
Every message $\mu_i$ or $\gamma_j$ flowing through an edge $(i,j)$ is multiplied by the edge weight $w_{ij}$.

\begin{algorithm}[ht]
  \caption{ChoiceRank for the weighted model}
  \label{cr:alg:wchoicerank}
  \begin{algorithmic}[1]
    \Require graph $\mathcal{G} = (\mathcal{V}, \mathcal{E})$, counts $\{ (c^-_i, c^+_i) \}$
    \State $\bm{\gamma} \gets [1 \ \cdots \ 1]^\Tr$
    \Repeat
      \State $\bm{z} \gets \bm{0}_N$
      \Comment{Recompute $\bm{\mu}$}
      \OneLineFor{$(i, j) \in \mathcal{E}$} $z_i \gets z_i + w_{ij} \gamma_j$ \label{cr:line:msg1}
      \OneLineFor{$i \in \mathcal{V}$} $\mu_i \gets c^+_i / z_i$
      \State $\bm{z} \gets \bm{0}_N$
      \Comment{Recompute $\bm{\gamma}$}
      \OneLineFor{$(i, j) \in \mathcal{E}$} $z_j \gets z_j + w_{ij} \mu_i$ \label{cr:line:msg2}
      \OneLineFor{$i \in \mathcal{V}$} $\gamma_i \gets (c^-_i + \alpha - 1) / (z_i + \beta)$
    \Until $\bm{\gamma}$ has converged
  \end{algorithmic}
\end{algorithm}


%%%%%%%%%%%%%%%%%%%%%%%%%%%%%%%%%%%%%%%%%%%%%%%%%%%%%%%%%%%%%%%%%%%%%%%%%
\subsection{EM Viewpoint}

The MM algorithm can be seen from an EM viewpoint, following the ideas of \citet{caron2012efficient}.
We introduce $N$ independent random variables $\mathcal{Z} = \{ Z_i : i = 1, \ldots, N \}$, where
\begin{align*}
Z_i \sim \text{Gamma} \bigg( c^+_i, \sum_{j \in \mathcal{N}^+_i} w_{ij} \gamma_j \bigg).
\end{align*}
With the addition of these latent random variables the complete log-likelihood becomes
\begin{align*}
\ell(\bm{\gamma} ; \mathcal{D}, \mathcal{Z})
    &= \ell(\bm{\gamma}, \mathcal{D}) + \sum_{i = 1}^N \log p(z_i \mid \mathcal{D}, \bm{\gamma}) \\
    &= \sum_{i = 1}^N \bigg[ c^-_i \log \gamma_i - c^+_i \log \sum_{k \in \mathcal{N}^+_i} w_{ik} \gamma_k \bigg] \\
    &\qquad +\sum_{i = 1}^N \bigg[  c^+_i \log \sum_{k \in \mathcal{N}^+_i} w_{ik} \gamma_k - z_i \sum_{k \in \mathcal{N}^+_i} w_{ik} \gamma_k \bigg] + \kappa_6 \\
    &= \sum_{i = 1}^N \bigg[ c^-_i \log \gamma_i - z_i \sum_{k \in \mathcal{N}^+_i} w_{ik} \gamma_k \bigg] + \kappa_6.
\end{align*}
Using a $\text{Gamma}(\alpha, \beta)$ prior for each parameter, the expected value of the log-posterior with respect to the conditional $\mathcal{Z} \mid \mathcal{D}$ under the estimate $\bm{\gamma}^{(t)}$ is
\begin{align*}
Q(\bm{\gamma}, \bm{\gamma}^{(t)})
    &= \mathbf{E}_{\mathcal{Z} \mid \mathcal{D}, \bm{\gamma}^{(t)}} \left[ \ell(\bm{\gamma} ; \mathcal{D}, \mathcal{Z}) \right]
       + \log p(\bm{\gamma}) \\
    &=\sum_{i = 1}^N \bigg[ c^-_i \log \gamma_i - c^+_i \frac{\sum_{k \in \mathcal{N}^+_i} w_{ik} \gamma_k}{\sum_{k \in \mathcal{N}^+_i} w_{ik} \gamma^{(t)}_k} \bigg]
      + \sum_{i = 1}^N \bigg[ (\alpha -1) \log \gamma_i - \beta \gamma_i \bigg] + \kappa_7
\end{align*}
The EM algorithm starts with an initial $\bm{\gamma}^{(0)}$ and iteratively refines the estimate by solving the optimization problem $\bm{\gamma}^{(t+1)} = \Argmax_{\bm{\gamma}} Q(\bm{\gamma}, \bm{\gamma}^{(t)})$.
It is not difficult to see that for a given $\bm{\gamma}^{(t)}$, maximizing $Q(\bm{\gamma}, \bm{\gamma}^{(t)})$ is equivalent to maximizing the minorizing function $f^{(t)}(\bm{\gamma})$ defined in~\eqref{cr:eq:wminorizing}.
Hence, the MM and the EM viewpoint lead to the exact same sequence of iterates.

The EM formulation leads to a Gibbs sampler in a relatively straightforward way \citep{caron2012efficient}.
We leave a systematic treatment of Bayesian inference in the network choice model for future work.
