\chapter{Choices in Networks}
\label{ch:choicerank}

In this chapter\footnote{%
This chapter is based on \citet{maystre2017choicerank}.},
we address the problem of understanding how users navigate in a network of $N$ nodes.
We consider a setting where only aggregate node-level traffic is observed and tackle the task of learning edge transition probabilities.
We cast it as a preference learning problem and study a model where choices follow a variant of Luce's axiom.
In this case, the $\BigO{N}$ marginal counts of node visits are a sufficient statistic for the $\BigO{N^2}$ transition probabilities.
We show how to make the inference problem well-posed regardless of the network's structure, and we develop an iterative algorithm that scales to networks that contain billions of nodes and edges.
We apply the model to two clickstream datasets and show that it successfully recovers the transition probabilities using only the network structure and marginal (node-level) traffic data.
Finally, we also consider an application to mobility networks and apply the model to one year of rides on New York City's bicycle-sharing system.

%%%%%%%%%%%%%%%%%%%%%%%%%%%%%%%%%%%%%%%%%%%%%%%%%%%%%%%%%%%%%%%%%%%%%%%%%
\section{Introduction}  %%%%%%%%%%%%%%%%%%%%%%%%%%%%%%%%%%%%%%%%%%%%%%%%%
\label{sec:intro}

The problem of recovering a ranking over $n$ items from noisy outcomes of pairwise comparisons has attracted, in the last century, much research interest, driven by applications in sports \citep{elo1978rating}, social sciences \citep{thurstone1927law, salganik2015wiki} and---more recently---recommender systems \citep{houlsby2012collaborative}.
Whereas pairwise comparison models and related inference algorithms have been extensively studied, the issue of \emph{which pairwise comparisons to sample}, also known as active learning, has received significantly less attention.
To understand the potential benefits of adaptively selecting samples, consider the case where comparison outcomes are noiseless, i.e., consistent with a linear order on a set of $n$ items.
If pairs of items are selected at random, it is necessary to collect \BigOmega{n^2} comparisons to recover the ranking \citep{alon1994linear}.
In contrast, by using an efficient sorting algorithm, \BigO{n \log n} adaptively chosen comparisons are sufficient.
In this work, we demonstrate that sorting algorithms can also be helpful in the \emph{noisy} setting, where some comparison outcomes are inconsistent with the ranking: despite errors, sorting algorithms tend to select informative samples.
We focus on the Bradley--Terry (BT) model, a widely-used probabilistic model of comparison outcomes.
In this model, each item is associated with a parameter on the real line, and the probability of observing an incorrect outcome decreases as the distance between the items' parameters increases.

First, we study the output of a single execution of Quicksort when comparison outcomes are generated from a BT model, under the assumption that the distance between adjacent parameters is (stochastically) uniform across the ranking.
We measure the quality of a ranking estimate by its displacement with respect to the ground truth, i.e., the sum of rank differences.
We show that Quicksort's output is a good approximation to the ground-truth ranking: no method comparing every pair of items at most once can do better (up to constant factors).
Furthermore, we show that by aggregating \BigO{\log^5 n} independent runs of Quicksort, it is possible to recover the exact rank for all but a vanishing fraction of the items.
These theoretical results suggest that adaptive sampling is able to bring a substantial acceleration to the learning process.

Second, we propose a practical active-learning (AL) strategy that consists of repeatedly sorting the items.
We evaluate our sorting-based method on three datasets and compare it to existing AL methods.
We observe that \emph{all} the strategies that we consider lead to better ranking estimates noticeably faster than random sampling.
However, most strategies are challenging to operate and computationally expensive, thus hindering wider adoption \citep{schein2007active}.
In this regard, sorting-based AL stands out, as
\begin{enuminline}
\item it is computationally-speaking as inexpensive as random sampling, 
\item it is trivial to implement, and
\item it requires no tuning of hyperparameters.
\end{enuminline}

\subsection{Preliminaries and Notation}

We consider $n$ items that are represented by consecutive integers $[n] = \{1, \ldots, n\}$.
Without loss of generality, we assume that the items are ranked by increasing preference\footnote{
This convention greatly simplifies the notation throughout the paper, but differs from that used in most of the preference learning literature.
In our paper, the item with rank $1$ is the \emph{worst}.}, i.e., $i < j$ means that $j$ is (in expectation) preferred to $i$.
When $j$ is preferred to $i$ as a result of a pairwise comparison, we denote the observation by $i \prec j$.
If $i < j$, we say that $i \prec j$ is a \emph{consistent} outcome and $j \prec i$ an \emph{inconsistent} (incorrect) outcome.
In most of the paper, pairwise comparison outcomes follow a Bradley--Terry model with parameters $\bm{\theta} = \begin{bmatrix} \theta_1 & \cdots & \theta_n \end{bmatrix} \in \Set{R}^n$, denoted $\BT(\bm{\theta})$.
The parameters $\theta_1 < \cdots < \theta_n$ represent the utilities of items $1, \ldots, n$, and the probability of observing the outcome $i \prec j$ is
\begin{align*}
p(i \prec j \mid \bm{\theta}) = \frac{1}{1 + \exp[-(\theta_j - \theta_i)]}.
\end{align*}
The probability of observing an inconsistent comparison decreases with the distance between the items.
This captures the intuitive notion that some pairs of items are easy to compare and some are more difficult \citep{zermelo1928berechnung, bradley1952rank}.

A ranking $\sigma$ is a function that maps an item to its rank, i.e., $\sigma(i) =$ rank of item $i$.
The (ground-truth) identity ranking is denoted by \id, i.e. $\id(i) = i$.
To measure the quality of a ranking $\sigma$ with respect to the ground-truth, we consider the \emph{displacement}
\begin{align*}
\Disp{\sigma} = \sum_{i=1}^n | \sigma(i) - i |,
\end{align*}
also known as Spearman's footrule distance.
Another metric widely used in practice is the Kendall--Tau distance, defined as
$K(\sigma) = \sum_{i < j} \Indic{\sigma(i) > \sigma(j)}$.
Both metrics are equivalent up to a factor of two\footnote{$\Disp{\sigma} / 2 \le K(\sigma) \le \Disp{\sigma}$ \citep{diaconis1977spearman}.}, such that bounds on \Disp{\sigma} also hold for $K(\sigma)$ up to constant factors.

Finally, we say that an event $A$ holds \emph{with high probability} if $\Prob{A} \to 1$ as $n \to \infty$.
For a random variable $X$ and a sequence of numbers $a_n$, we say that $X = \BigO{a_n}$ with high probability if $\Prob{\Abs{X} \le c a_n} \to 1$ as $n \to \infty$ for some constant $c$ that does not depend on $n$.

\paragraph{Outline of the paper.}
We begin by briefly reviewing related literature in Section~\ref{sec:relwork}.
Next, in Section~\ref{sec:theory}, we study the displacement of Quicksort's output under noisy comparisons.
In Section~\ref{sec:experiments}, we empirically evaluate several AL strategies on three datasets.
Finally, we conclude in Section~\ref{sec:conclusion}.

\section{Related Work}
\label{pk:sec:relwork}

Zermelo's \citeyear{zermelo1928berechnung} paper (discussed in Section~\ref{in:sec:btmodel}) presented the first statistical model of chess game outcomes.
His model, associated with a simple online stochastic gradient update rule, is known as the Elo rating system \citep{elo1978rating}.
This rating system is currently used by the World Chess Federation (FIDE) to rank chess players\footnote{See: \url{https://ratings.fide.com/}.} and by the International Federation of Football Association (FIFA) to rank women's national football teams\footnote{See: \url{http://www.fifa.com/fifa-world-ranking/procedure/women.html}.}, among others.

The model and related inference algorithms have been extended in various ways, e.g., by considering other types of outcomes \citep{rao1967ties, maher1982modelling} or by permitting parameters to evolve over time \citep{glickman1993paired, fahrmeir1994dynamic, cattelan2013dynamic}.
One direction that is of particular interest in this chapter is the handling of the uncertainty of the estimated skill parameters.
\citet{glickman1999parameter} proposes an extension that simultaneously updates ratings and associated uncertainty values, after each observation, by using a simple closed-form update.
\citet{herbrich2006trueskill} propose TrueSkill, a comprehensive Bayesian framework for estimating player skills in various types of games based on the expectation-propagation algorithm.
The models and methods described in this chapter are similar to TrueSkill, as will be discussed in Section~\ref{pk:sec:methods}.
In the context of learning users' preferences from pairwise comparisons, \citet{chu2005preference} were the first to link the Bayesian treatment of pairwise comparisons models to Gaussian-process classification.

%%%%%%%%%%%%%%%%%%%%%%%%%%%%%%%%%%%%%%%%%%%%%%%%%%%%%%%%%%%%%%%%%%%%%%%%%
\section{Network Choice Model}  %%%%%%%%%%%%%%%%%%%%%%%%%%%%%%%%%%%%%%%%%
\label{cr:sec:model}

Let $\mathcal{G} = (\mathcal{V}, \mathcal{E})$ be a directed graph on $N$ nodes (corresponding to items) and $M$ edges, with edge weights $w_{ij} > 0$ for all $(i, j) \in \mathcal{E}$.
We denote the out-neighborhood of node $i$ by $\mathcal{N}^+_i$ and its in-neighborhood by $\mathcal{N}^-_i$.
We consider the following choice process on $\mathcal{G}$.
A user starts at a node $i$ and is faced with alternatives $\mathcal{N}^+_i$.
The user chooses item $j$ and moves to the corresponding node.
At node $j$, the user is faced with alternatives $\mathcal{N}^+_j$ and chooses $k$, and so on.
At any time, the user can stop.
Figure~\ref{cr:fig:samplenet} gives an example of a graph and the alternatives available at a step of the process.

\begin{figure}
  \centering
  \includegraphics[scale=0.8]{cr-graph-example}
  \caption{An illustration of one step of the process.
  The user is at node 6 and can reach nodes $\mathcal{N}^+_6 = \{1, 2, 5, 7\}$.}
  \label{cr:fig:samplenet}
\end{figure}

To define the transition probabilities, we follow \citet{kumar2015inverting} and posit a probabilistic model of choice, which extends that of \citet{luce1959individual}.
For every node $i$ and every $j \in \mathcal{N}^+_i$, the probability that $j$ is selected among alternatives $\mathcal{N}^+_i$ can be written as
\begin{align}
\label{cr:eq:singlelik}
p_{ij} = \frac{w_{ij} \gamma_j}{\sum_{k \in \mathcal{N}^+_i} w_{ik} \gamma_k},
\end{align}
for some parameter vector $\bm{\gamma} = [\gamma_1 \ \cdots \ \gamma_N]^\Tr \in \mathbf{R}_{>0}^N$.
Intuitively, the parameter $\gamma_i$ can be interpreted as the utility of item $i$.
The edge weights are relevant in situations where the current context modulates the alternatives' utility;
for example, they can be used to encode the position or prominence of a link on a page in a hyperlink graph, or the distance between two locations in a mobility network.
Luce's original choice model is obtained by setting $w_{ij} \doteq \text{constant}$.
Note that $p_{ij}$ depends only on the out-neighborhood of node $i$.
As such, the choice process satisfies the Markov property, and we can think of the sequence of choices as a trajectory in a Markov chain.

In the context of this model, we can formulate the inference problem as follows.
Given a directed graph $\mathcal{G} = (\mathcal{V}, \mathcal{E})$, edge weights $\{ w_{ij} \}$ and data on the aggregate traffic at each node, find a parameter vector $\bm{\gamma}$ that fits the data.

\paragraph{Notation}
In some expressions, we use $\kappa$ to denote a constant that does not depend on the parameter vector $\bm{\gamma}$.
Its value can change from line to line.

%%%%%%%%%%%%%%%%%%%%%%%%%%%%%%%%%%%%%%%%%%%%%%%%%%%%%%%%%%%%%%%%%%%%%%%%%
\subsection{Sufficient Statistic}

We begin by showing that $\BigO{N}$ values summarizing the aggregate traffic at each node are a sufficient statistic of the transition counts.
Let $c_{ij}$ denote the number of transitions that occurred along edge $(i, j) \in \mathcal{E}$.
Starting from the transition probability defined in~\eqref{cr:eq:singlelik}, we can write the log-likelihood of $\bm{\gamma}$ given data $\mathcal{D} = \{ c_{ij} : (i, j) \in \mathcal{E} \}$ as
\begin{align}
\ell(\bm{\gamma} ; \mathcal{D})
    &= \sum_{(i,j) \in \mathcal{E}} c_{ij} \bigg[ \log w_{ij} \gamma_j - \log \sum_{k \in \mathcal{N}^+_i} w_{ik} \gamma_k \bigg] \nonumber \\
    &= \sum_{j = 1}^N \sum_{i \in \mathcal{N}^-_j}\!c_{ij} \log \gamma_j
       - \sum_{i = 1}^N \sum_{j \in \mathcal{N}^+_i}\!c_{ij} \log \sum_{k \in \mathcal{N}^+_i} w_{ik} \gamma_k
       + \sum_{(i,j) \in \mathcal{E}} c_{ij} \log w_{ij}, \nonumber \\
    &= \sum_{i = 1}^N \bigg[ c^-_i \log \gamma_i - c^+_i \log\!\sum_{k \in \mathcal{N}^+_i}\!w_{ik} \gamma_k \bigg] + \kappa, \label{cr:eq:loglik}
\end{align}
where $c^-_i = \sum_{j \in \mathcal{N}^-_i} c_{ji}$ and $c^+_i = \sum_{j \in \mathcal{N}^+_i} c_{ij}$ is the aggregate number of transitions arriving in and originating from $i$, respectively.
This formulation of the log-likelihood exhibits a key feature of the model:
the set of $2N$ counts $\{ (c^-_i, c^+_i) : i \in \mathcal{V} \}$ is a sufficient statistic of the $M = \BigO{N^2}$ counts $\{ c_{ij} : (i, j) \in \mathcal{E} \}$ for the parameters $\bm{\gamma}$.

\begin{theorem}
The set of aggregate transitions  $\{ (c^-_i, c^+_i) : i \in \mathcal{V} \}$ is a minimally sufficient statistic for the parameters $\bm{\gamma}$.
\end{theorem}

\begin{proof}
Let $f(\{ c_{ij} \} \mid \bm{\gamma})$ be the discrete probability density function of the data under the model with parameters $\bm{\gamma}$.
By Theorem $6.2.13$ in \citet{casella2002statistical}, $\{ (c^-_i, c^+_i) \}$ is a minimally sufficient statistic for $\bm{\gamma}$ if and only if, for any $\{ c_{ij} \}$ and $\{ d_{ij} \}$ in the support of $f$,
\begin{align}
\label{cr:eq:minsuff}
\begin{aligned}
\frac{ f(\{ c_{ij} \} \mid \bm{\gamma}) }{ f(\{ d_{ij} \} \mid \bm{\gamma}) }\ \text{is independent of $\bm{\gamma}$}
\iff (c^-_i, c^+_i) = (d^-_i, d^+_i) \quad \forall i.
\end{aligned}
\end{align}
Taking the log of the ratio on the left-hand side and using~\eqref{cr:eq:loglik}, we find that
\begin{align*}
\log \frac{ f(\{ c_{ij} \} \mid \bm{\gamma}) }{ f(\{ d_{ij} \} \mid \bm{\gamma}) } =
  \sum_{i = 1}^N \bigg[ (c^-_i\!-\!d^-_i) \log \gamma_i
                       - (c^+_i\!-\!d^+_i) \log\!\sum_{k \in \mathcal{N}^+_i}\!w_{ik} \gamma_k \bigg] + \kappa.
\end{align*}
From this, it is easy to see that the ratio of densities is independent of $\bm{\gamma}$ if and only if $c^-_i = d^-_i$ and $c^+_i = d^+_i$, which verifies~\eqref{cr:eq:minsuff}.
\end{proof}

In other words, it is enough to observe marginal information about the number of arrivals and departures at each node---we call this collective data the \emph{traffic} at a node---and no additional information can be gained by observing the full choice process.
This makes the model particularly attractive, because it means that it is unnecessary to track users across nodes.
In several applications of practical interest, tracking users is undesirable, difficult, or outright impossible, due to
\begin{enuminline}
\item privacy reasons,
\item monitoring costs, or
\item lack of data in existing datasets.
\end{enuminline}

Note that if we make the additional assumption that the flow in the network is conserved, then $c^-_i = c^+_i$.
If users' typical trajectories are made of many hops, it is reasonable to approximate $c^-_i$ or $c^+_i$ by using this assumption, should one of the two quantities be missing.

%%%%%%%%%%%%%%%%%%%%%%%%%%%%%%%%%%%%%%%%%%%%%%%%%%%%%%%%%%%%%%%%%%%%%%%%%
\subsection{Steady-State Inversion Problem}
% Also works if all the Markov trajectories are loops.

In recent work, \citet{kumar2015inverting} define the problem of \emph{steady-state inversion} as follows:
Given a strongly-connected directed graph $\mathcal{G} = (\mathcal{V}, \mathcal{E})$ with edge weights $\{ w_{ij} \}$ and a target distribution over the nodes $\bm{\pi}$, find the transition matrix of a Markov chain on $\mathcal{G}$ with stationary distribution $\bm{\pi}$.
As there are $M = \BigO{N^2}$ degrees of freedom (the transition probabilities) for $N$ constraints (the stationary distribution), the problem is in most cases underdetermined.
Following Luce's ideas, the transition probabilities are constrained to be proportional to a latent score of the destination node as per \eqref{cr:eq:singlelik}, thus reducing the number of parameters from $M$ to $N$.
Denote by $\bm{P}(\bm{s})$ the Markov-chain transition matrix parametrized with scores $\bm{s}$.
The score vector $\bm{s}$ is a solution for the steady-state inversion problem if and only if $\bm{\pi} = \bm{\pi}^\Tr \bm{P}(\bm{s})$, or equivalently
\begin{align}
\label{cr:eq:balance}
\pi_i = \sum_{j \in \mathcal{N}^-_i} \frac{w_{ji} s_i}{\sum_{k \in \mathcal{N}^+_j} w_{jk} s_k} \pi_j \quad \forall i.
\end{align}
In order to formalize the connection between \citeauthor{kumar2015inverting}'s work and ours, we express the steady-state inversion problem as that of asymptotic maximum-likelihood estimation in the network choice model.
Suppose that we observe node-level traffic data $\mathcal{D} = \{ (c^-_i, c^+_i) : i \in \mathcal{V} \}$ about a trajectory of length $T$ starting at an arbitrary node.
We want to obtain an estimate of the parameters $\bm{\gamma}^\star$ by maximizing the average log-likelihood $\hat{\ell}(\bm{\gamma}) = \frac{1}{T} \ell (\bm{\gamma} ; \mathcal{D})$.
From standard convergence results for Markov chains \citep{kemeny1976finite}, it follows that as $\mathcal{G}$ is strongly connected, $\lim_{T \to \infty} c^-_i / T = \lim_{T \to \infty} c^+_i / T = \pi_i$.
Therefore,
\begin{align*}
\hat{\ell}(\bm{\gamma})
    = \sum_{i = 1}^N \bigg[ \frac{c^-_i}{T} \log \gamma_i - \frac{c^+_i}{T} \log \sum_{k \in \mathcal{N}^+_i} w_{ik} \gamma_k \bigg]
    \xrightarrow{T \to \infty} \sum_{i = 1}^N \pi_i \bigg[ \log \gamma_i - \log \sum_{k \in \mathcal{N}^+_i} w_{ik} \gamma_k \bigg].
\end{align*}
Let $\bm{\gamma}^\star$ be a maximizer of the average log-likelihood.
When $T \to \infty$, the optimality condition $\nabla \hat{\ell} = \bm{0}$ implies, for all $i$,
\begin{align}
&\frac{\partial \hat{\ell}(\bm{\gamma})}{\partial \gamma_i} \bigg|_{\bm{\gamma} = \bm{\gamma}^\star}
    = \frac{\pi_i}{\gamma^\star_i} - \sum_{j \in \mathcal{N}^-_i} \frac{w_{ji} \pi_j}{\sum_{k \in \mathcal{N}^+_j} w_{jk} \gamma^\star_k}
    = 0 \nonumber \\
&\qquad \iff \pi_i = \sum_{j \in \mathcal{N}^-_i} \frac{w_{ji} \gamma^\star_i}{\sum_{k \in \mathcal{N}^+_j} w_{jk} \gamma^\star_k} \pi_j. \label{cr:eq:optimality}
\end{align}
Comparing~\eqref{cr:eq:optimality} to~\eqref{cr:eq:balance}, it is clear that $\bm{\gamma}^\star$ is a solution of the steady-state inversion problem.
As such, the network choice model presented in this chapter can be viewed as a principled extension of the steady-state inversion problem to the finite-data case.


%%%%%%%%%%%%%%%%%%%%%%%%%%%%%%%%%%%%%%%%%%%%%%%%%%%%%%%%%%%%%%%%%%%%%%%%%
\subsection{MLE}
\label{cr:sec:maxlik}

The log-likelihood~\eqref{cr:eq:loglik} is not concave in $\bm{\gamma}$, but it can be made concave by using the standard reparametrization $\gamma_i = e^{\theta_i}$.
Therefore, any local minimum of the likelihood is a global minimum (c.f. Section~\ref{fi:sec:mle}).
Unfortunately, it turns out that the conditions guaranteeing that the ML estimate is well-defined (i.e., that it exists and is unique) are restrictive and impractical.

\begin{definition}[comparison graph]
Let $\mathcal{G} = (\mathcal{V}, \mathcal{E})$ be a directed graph and $\{ a_{ij} : (i,j) \in \mathcal{E} \}$ be non-negative numbers.
The \emph{comparison graph} induced by $\{ a_{ij} \}$ is the directed graph $\mathcal{G}' = (\mathcal{V}, \mathcal{E}')$, where $(i,j) \in \mathcal{E}'$ if and only if there is a node $k$ such that $i, j \in \mathcal{N}^+_k$ and $a_{kj} > 0$.
\end{definition}

The numbers $\{ a_{ij}\}$ in the definition can be loosely interpreted as transition counts (although they do not need to be integers).
Intuitively, there is an edge $(i, j)$ in the comparison graph whenever there is at least one instance in which $i$ and $j$ are among the alternatives and $j$ is selected.
The notion of comparison graph leads to a precise characterization of whether the ML estimate is well-defined or not, as shown by the next theorem---an extension of Theorem~\ref{fi:thm:mlboth} to the network choice model.

\begin{theorem}
\label{cr:thm:mlboth}
Let $\mathcal{G} = (\mathcal{V}, \mathcal{E})$ be a weighted, directed graph and $\{ (c^-_i, c^+_i) \}$ be the aggregate number of transitions arriving in and originating from $i$, respectively.
Let $\{ a_{ij} \}$ be any set of non-negative real numbers that satisfy
\begin{align*}
\sum_{j \in \mathcal{N}^-_i} a_{ji} = c^-_i, \quad
\sum_{j \in \mathcal{N}^+_i} a_{ij} = c^+_i \quad \forall i.
\end{align*}
Then, the maximizer of the log-likelihood~\eqref{cr:eq:loglik} exists and is unique (up to rescaling) if and only if the comparison graph induced by $\{ a_{ij} \}$ is strongly connected.
\end{theorem}

\begin{proof}
The proof borrows from \citet{hunter2004mm}, in particular from the proofs of Lemmas~$1$ and~$2$.
Using $\gamma_i = e^{\theta_i}$, we can rewrite the reparametrized log-likelihood using $\{ a_{ij} \}$ as
\begin{align*}
    \ell(\bm{\theta})
        = \sum_{i = 1}^N \sum_{j \in \mathcal{N}^+_i} a_{ij} \bigg[ \theta_j - \log \sum_{k \in \mathcal{N}^+_i} w_{ik} e^{\theta_k} \bigg],
\end{align*}
and, without loss of generality, we can assume that $\sum_i \theta_i = 0$ and $\min_{ij} w_{ij} = 1$.
We study the conditions under which
\begin{enuminline}
\item super-level sets of the likelihood function $\ell(\bm{\theta})$ are bounded, and
\item the likelihood function is strictly concave.
\end{enuminline}

First, we prove that the super-level set $\{ \bm{\theta} : \ell(\bm{\theta}) \ge c \}$ is bounded and compact for any $c$, if and only if the comparison graph is strongly connected.
The compactness of all super-level sets ensures that there is at least one maximizer.
Pick any unit-norm vector $\bm{u}$ such that $\sum_i u_i = 0$, and let $\bm{\theta} = s \bm{u}$
When $s \to \infty$, then $e^{\theta_i} > 0$ and $e^{\theta_j} \to 0 $ for some $i$ and $j$.
As the comparison graph is strongly connected, there is a path from $i$ to $j$, and along this path there must be two consecutive nodes $i', j'$ such that $e^{\theta_{i'}} > 0$ and $e^{\theta_{j'}} \to 0$.
The existence of the edge $(i',j')$ in the comparison graph means that there is a $k$ such that $i', j' \in \mathcal{N}^+_k$ and $a_{kj'} > 0$.
Therefore, the log-likelihood can be bounded as
\begin{align*}
\ell(\bm{\theta})
    \le a_{kj'} \bigg[ \theta_{j'} - \log \sum_{q \in \mathcal{N}^+_k} w_{kq} e^{\theta_q} \bigg]
    \le a_{kj'} \left[ \theta_{j'} - \log (e^{\theta_{j'}} + e^{\theta_{i'}}) \right],
\end{align*}
and $\lim_{s \to \infty} \ell(\bm{\theta}) = -\infty$.
Conversely, suppose that the comparison graph is not strongly connected and partition the vertices into two non-empty subsets $\mathcal{S}$ and $\mathcal{T}$ such that there is no edge from $\mathcal{S}$ to $\mathcal{T}$.
Let $c > 0$ be any positive constant, and take $\tilde{\theta}_i = \theta_i + c$ if $i \in \mathcal{S}$ and $\tilde{\theta}_i = \theta_i$ if $i \in \mathcal{T}$ (renormalize such that $\sum_i \tilde{\theta}_i = 0$).
Clearly, $\ell(\tilde{\bm{\theta}}) \ge \ell(\bm{\theta})$, and, by repeating this procedure, $\Norm{\bm{\theta}}$ can be driven to infinity without decreasing the likelihood.

Second, we prove that if the comparison graph is strongly connected, the log-likelihood is strictly concave (in $\bm{\theta}$).
In particular, for any $p \in (0,1)$,
\begin{align}
\label{cr:eq:strictconcav}
\ell \left[ p \bm{\theta} + (1-p) \bm{\eta} \right] \ge p \ell(\bm{\theta}) + (1-p) \ell(\bm{\eta}),
\end{align}
with equality if and only if $\bm{\theta} \equiv \bm{\eta}$ up to a constant shift.
Strict concavity ensures that there is at most one maximizer of log-likelihood.
We start with Hölder's inequality, which implies that, for positive $\{ x_k \}$ and $\{ y_k \}$, and $p \in (0,1)$,
\begin{align*}
\log \sum_k x_k^p y_k^{1-p} \le p \log \sum_k x_k + (1-p) \log \sum_k y_k.
\end{align*}
with equality if and only $x_k = c y_k$ for some $c > 0$.
Letting $x_k = w_{ik} e^{\theta_k}$ and $y_k = w_{ik} e^{\eta_k}$, we find that for all $i$
\begin{align}
\label{cr:eq:holderapp}
\begin{aligned}
\log \sum_{k \in \mathcal{N}^+_i} w_{ik} e^{p \theta_k + (1-p) \eta_k}
    \le p \log\!\sum_{k \in \mathcal{N}^+_i}\!w_{ik} e^{\theta_k} + (1-p) \log\!\sum_{k \in \mathcal{N}^+_i}\!w_{ik} e^{\eta_k},
\end{aligned}
\end{align}
with equality if and only if there exists $c \in \mathbf{R}$ such that $\theta_k = \eta_k + c$ for all $k \in \mathcal{N}^+_{i}$.
Multiplying by $a_{ij}$ and summing over $i$ and $j$ on both sides of~\eqref{cr:eq:holderapp} shows that the log-likelihood is concave in $\bm{\theta}$.
Now, consider any partition of the vertices into two non-empty subsets $\mathcal{S}$ and $\mathcal{T}$.
Because the comparison graph is strongly connected, there is always $k \in \mathcal{V}$, $i \in \mathcal{S}$ and $j \in \mathcal{T}$ such that $i, j \in \mathcal{N}^+_k$ and $a_{ki} > 0$.
Therefore, the left and right side of~\eqref{cr:eq:strictconcav} are equal if and only if $\bm{\theta} \equiv \bm{\eta}$ up to a constant shift.
\end{proof}

In order to verify the necessary and sufficient condition of Theorem~\ref{cr:thm:mlboth} given $\{ (c^-_i, c^+_i) \}$, we have to find a non-negative solution $\{ a_{ij} \}$ to the system of equations
\begin{align*}
\sum_{j \in \mathcal{N}^-_i} a_{ji} = c^-_i, \quad
\sum_{j \in \mathcal{N}^+_i} a_{ij} = c^+_i \quad \forall i.
\end{align*}
\citet{dines1926positive} presents a simple algorithm to find such a non-negative solution.
Alternatively, \citet{kumar2015inverting} suggest recasting the problem as one of maximum flow in a network.
However, the computational cost of running \citeauthor{dines1926positive}' or max-flow algorithms is significantly greater than that of running the inference algorithm that we develop later, in Section~\ref{cr:sec:algorithm}.

\paragraph{Example}
In order to illustrate Theorem~\ref{cr:thm:mlboth}, we describe an innocuous-looking example where the MLE does not exist.
Consider the network structure and traffic data depicted in Figure~\ref{cr:fig:badexample}.
The network is strongly connected and every node $i$ has positive incoming and outgoing traffic $c^-_i$ and $c^+_i$.
Nevertheless, the corresponding comparison graph is \emph{not} strongly connected, and it turns out that the likelihood can be made arbitrarily large by increasing $\gamma_1$, $\gamma_2$ and $\gamma_4$.
In this simple example, we indicate the edge transitions that generated the observed marginal traffic in bold.
Given this information, the comparison graph is easy to find, and the necessary and sufficient condition is easy to check.
But in general, finding a set of transitions that is compatible with given marginal per-node traffic data is a nontrivial computation.

\begin{figure}
  \begin{subfigure}{.49\textwidth}
    \centering
    \includegraphics[width=.85\linewidth]{cr-ml-undefined}
    \caption{network structure}
  \end{subfigure}%
  \begin{subfigure}{.49\textwidth}
    \centering
    \includegraphics[width=.85\linewidth]{cr-ml-undefined-comp}
    \caption{comparison graph}
  \end{subfigure}
  \caption{An innocent-looking example where the ML estimate does not exist.
  The network structure, aggregate traffic data and compatible transitions are shown on the left.
  The comparison graph (right) is not strongly connected.}
  \label{cr:fig:badexample}
\end{figure}

\paragraph{Necessary Condition}
As the conditions of Theorem~\ref{cr:thm:mlboth} involve the observed traffic, we might ask the following question.
Is there a simpler condition on the structure of $\mathcal{G}$ such that the MLE is well-defined, given sufficiently many transitions?
We provide an answer in the form of a necessary condition for the uniqueness of the MLE that involves only the structure of the network.
We begin with a definition that uses the notion of \emph{hypergraph}, a generalized graph where edges can be any non-empty subset of nodes.

\begin{definition}[alternatives hypergraph]
Given a directed graph $\mathcal{G} = (\mathcal{V}, \mathcal{E})$, the \emph{alternatives hypergraph} is defined as $\mathcal{H} = (\mathcal{V}, \mathcal{A})$, with $\mathcal{A} = \{ \mathcal{N}^+_i : i \in \mathcal{V} \}$.
\end{definition}

Intuitively, $\mathcal{H}$ is the hypergraph induced by the sets of alternatives available at each node.
Equipped with this definition, we can state the following corollary of Theorem~\ref{cr:thm:mlboth}.

\begin{corollary}
\label{cr:thm:mlnecessary}
If the alternatives hypergraph is not connected, then for any data $\mathcal{D}$ there are $\bm{\gamma}$ and $\bm{\lambda}$ such that $\bm{\gamma} \neq c \bm{\lambda}$ for any $c \in \mathbf{R}_{>0}$ and $\ell(\bm{\gamma} ; \mathcal{D}) = \ell(\bm{\lambda} ; \mathcal{D}).$
\end{corollary}

\begin{proof}
If the alternatives hypergraph is disconnected, then for any data $\mathcal{D}$, the comparison graph is disconnected too.
Furthermore, the connected components of the comparison graph are subsets of those of the hypergraph.
Partition the vertices into two non-empty subsets $\mathcal{S}$ and $\mathcal{T}$ such that there is no hyperedge between $\mathcal{S}$ to $\mathcal{T}$ in the alternatives hypergraph.
Let $\mathcal{A} = \{ i : \mathcal{N}^+_i \subset \mathcal{S} \}$ and $\mathcal{B} = \{ i : \mathcal{N}^+_i \subset \mathcal{T} \}$.
By construction of the alternatives hypergraph, $\mathcal{A} \cap \mathcal{B} = \varnothing$ and $\mathcal{A} \cup \mathcal{B} = \mathcal{V}$.
The log-likelihood can be rewritten as
\begin{align*}
\ell(\bm{\theta}) =
    &\sum_{i \in \mathcal{A}} \sum_{j \in \mathcal{N}^+_i} a_{ij}
        \bigg[ \log \gamma_j - \log \sum_{k \in \mathcal{N}^+_i} w_{ik} \gamma_k \bigg] \\
    &+ \sum_{i \in \mathcal{B}} \sum_{j \in \mathcal{N}^+_i} a_{ij}
        \bigg[ \log \gamma_j - \log \sum_{k \in \mathcal{N}^+_i} w_{ik} \gamma_k \bigg].
\end{align*}
The sum over $\mathcal{A}$ involves only parameters related to nodes in $\mathcal{S}$, whereas the sum over $\mathcal{B}$ involves only parameters related to nodes in $\mathcal{T}$.
Because the likelihood is invariant to a rescaling of the parameters, it is easy to see that we can arbitrarily rescale the parameters of the vertices in either $\mathcal{S}$ or $\mathcal{T}$ without affecting the likelihood.
\end{proof}

The network of Figure~\ref{cr:fig:samplenet} illustrates an instance where even the necessary the condition fails:
although the graph $\mathcal{G}$ is strongly connected, its associated alternatives hypergraph $\mathcal{H}$ (depicted in Figure~\ref{cr:fig:samplehyp}) is disconnected, and no matter what the data $\mathcal{D}$ is, the ML estimate will never be uniquely defined.
Note that this problematic situation does not affect only carefully hand-crafted networks: the alternatives hypergraph of all three real-world networks considered in Section~\ref{cr:sec:experiments} are disconnected as well.

\begin{figure}
  \centering
  \includegraphics[width=.5\linewidth]{cr-graph-example-pref}
  \caption{The alternatives hypergraph associated with the network of Figure~\ref{cr:fig:samplenet}.
The hyperedge associated with $\mathcal{N}^+_6$ is highlighted in red.
Note that the component $\{3, 4\}$ is disconnected from the rest of the hypergraph.}
  \label{cr:fig:samplehyp}
\end{figure}

%%%%%%%%%%%%%%%%%%%%%%%%%%%%%%%%%%%%%%%%%%%%%%%%%%%%%%%%%%%%%%%%%%%%%%%%%
\section{Well-Posed Inference} %%%%%%%%%%%%%%%%%%%%%%%%%%%%%%%%%%%%%%%%%%
\label{cr:sec:inference}
% https://en.wikipedia.org/wiki/Well-posed_problem

The shortcomings of the MLE discussed in the previous section drive us to seek a more robust estimator.
Following the ideas of \citet{caron2012efficient}, we introduce an independent Gamma prior on each parameter, i.e., i.i.d. $\gamma_1, \ldots, \gamma_N \sim \DGamma{\alpha, \beta}$.
Adding the log-prior to the log-likelihood, we can write the log-posterior as
\begin{align}
\label{cr:eq:logpost}
\log p(\bm{\gamma} \mid \mathcal{D}) =
    \sum_{i = 1}^N \bigg[ (c^-_i + \alpha - 1) \log \gamma_i
        - c^+_i \log \bigg( \sum_{k \in \mathcal{N}^+_i} w_{ik} \gamma_k \bigg)  - \beta \gamma_i \bigg]
    + \kappa.
\end{align}
The Gamma prior translates into a form of regularization that makes the inference problem well-posed, as shown by the following theorem.

\begin{theorem}
\label{cr:thm:map}
If i.i.d. $\gamma_1, \ldots, \gamma_N \sim \DGamma{\alpha, \beta}$ with $\alpha > 1$, then the log-posterior~\eqref{cr:eq:logpost} always has a unique maximizer $\bm{\gamma}^\star \in \mathbf{R}^N_{>0}$.
\end{theorem}

\begin{proof}
Under the reparametrization $\gamma_i = e^{\theta_i}$, the log-prior and the log-likelihood become
\begin{align*}
\log p(\bm{\theta})
    &= \sum_{i = 1}^N \left[ (\alpha - 1) \theta_i - \beta e^{\theta_i} \right] + \kappa \\
\ell(\bm{\theta} ; \mathcal{D})
    &= \sum_{i = 1}^N \bigg[ c^-_i \theta_i - c^+_i \log \sum_{k \in \mathcal{N}^+_i} w_{ik} e^{\theta_k} \bigg] + \kappa.
\end{align*}
It is easy to see that the log-likelihood is concave and the log-prior strictly concave in $\bm{\theta}$ (for $\alpha > 1$).
As a result, the log-posterior is strictly concave in $\bm{\theta}$, which ensures that there exists at most one maximizer.

Now consider any transition counts $\{ c_{ij} \}$ that satisfy $c^-_i = \sum_{j \in \mathcal{N}^-_i} c_{ji}$ and $c^+_i = \sum_{j \in \mathcal{N}^+_i} c_{ij}$.
The log-posterior can be written as
\begin{align*}
\log p(\bm{\theta} \mid \mathcal{D})
    &= \sum_{i = 1}^N \sum_{j \in \mathcal{N}^+_i} c_{ij} \bigg[ \theta_j - \log \sum_{k \in \mathcal{N}^+_i} w_{ik} e^{\theta_k} \bigg]
       + \sum_{i = 1}^N \left[ (\alpha - 1) \theta_i - \beta e^{\theta_i} \right] + \kappa\\
    &\le -N^2 \cdot \max_{i,j} \log w_{ij}
       + \sum_{i = 1}^N \left[ (\alpha - 1) \theta_i - \beta e^{\theta_i} \right] + \kappa.
\end{align*}
For $\alpha > 1$, it follows that $\lim_{\Norm{\bm{\theta}} \to \infty} \log p(\bm{\theta} \mid \mathcal{D}) = -\infty$, which ensures that there is at least one maximizer.
\end{proof}

Note that varying the rate $\beta$ in the Gamma prior simply rescales the parameters $\bm{\gamma}$.
Furthermore, it is clear from~\eqref{cr:eq:singlelik} that such a rescaling affects neither the likelihood of the observed data nor the prediction of future transitions.
As a consequence, we may assume that $\beta = 1$ without loss of generality.

%%%%%%%%%%%%%%%%%%%%%%%%%%%%%%%%%%%%%%%%%%%%%%%%%%%%%%%%%%%%%%%%%%
\subsection{ChoiceRank Algorithm}  %%%%%%%%%%%%%%%%%%%%%%%%%%%%%%%%%%%%%%%%%%
\label{cr:sec:algorithm}

The maximizer of the log-posterior does not have a closed-form solution.
In the spirit of the algorithms of \citet{hunter2004mm} for variants of Luce's choice model, we develop a minorization-maximization (MM) algorithm.
Simply put, the algorithm iteratively refines an estimate of the maximizer by solving a sequence of simpler optimization problems.
Using the inequality $\log x \le \log \tilde{x} + x/\tilde{x} - 1$ (with equality if and only if $x = \tilde{x}$), we can lower-bound the log-posterior~\eqref{cr:eq:logpost} by
\begin{align}
\label{cr:eq:minorizing}
\begin{aligned}
f^{(t)}(\bm{\gamma}) = \kappa + \sum_{i = 1}^N \bigg[
    & (c^-_i + \alpha - 1) \log \gamma_i - \beta \gamma_i \\
    &- c^+_i \bigg( \log\!\sum_{k \in \mathcal{N}^+_i}\!w_{ik} \gamma^{(t)}_k
                   +\frac{\sum_{k \in \mathcal{N}^+_i}\!w_{ik} \gamma_k}{\sum_{k \in \mathcal{N}^+_i}\!w_{ik} \gamma^{(t)}_k} -1 \bigg) \bigg],
\end{aligned}
\end{align}
with equality if and only if $\bm{\gamma} = \bm{\gamma}^{(t)}$.
Starting with an arbitrary $\bm{\gamma}^{(0)} \in \mathbf{R}^N_{>0}$, we repeatedly solve the optimization problem
\begin{align*}
\bm{\gamma}^{(t+1)} = \Argmax_{\bm{\gamma}} f^{(t)}(\bm{\gamma}).
\end{align*}
Unlike the maximization of the log-posterior, the surrogate optimization problem has a closed-form solution, obtained by setting $\nabla f^{(t)}$ to $\bm{0}$:
\begin{align}
\label{cr:eq:mmiter}
\gamma_i^{(t + 1)} = \frac{c^-_i + \alpha - 1}{\sum_{j \in \mathcal{N}^-_i} w_{ji} \mu_j^{(t)} + \beta},
    \quad \text{where }
    \mu_j^{(t)} = \frac{c^+_j}{\sum_{k \in \mathcal{N}^+_j} w_{jk} \gamma_k^{(t)}}.
\end{align}
The sequence of iterates provably converges to the maximizer of the log-posterior~\eqref{cr:eq:logpost}, as shown by the following theorem.

\begin{theorem}
\label{cr:thm:mmconv}
Let $\bm{\gamma}^\star$ be the unique maximum a-posteriori estimate.
Then for any initial $\bm{\gamma}^{(0)} \in \mathbf{R}^N_{> 0}$ the sequence of iterates defined by~\eqref{cr:eq:mmiter} converges to $\bm{\gamma}^\star$.
\end{theorem}

\begin{proof}
The proof follows that of Theorem~$1$ in \citet{hunter2004mm}.
Let $M: \mathbf{R}^N_{>0} \to \mathbf{R}^N_{>0}$ be the (continuous) map implicitly defined by one iteration of the algorithm.
For conciseness, let $g(\bm{\gamma}) \doteq \log p(\bm{\gamma} \mid \mathcal{D})$.
As $g$ has a unique maximizer and is concave using the reparametrization $\gamma_i = e^{\theta_i}$, it follows that $g$ has a single stationary point.
First, observe that the minorization-maximization property guarantees that $g \left[ M(\bm{\gamma}) \right] \ge g(\bm{\gamma})$.
Combined with the strict concavity of $g$, this ensures that $\lim_{t \to \infty} g(\bm{\gamma}^{(t)})$ exists and is unique for any $\bm{\gamma}^{(0)}$.
Second, $g \left[ M(\bm{\gamma}) \right] = g(\bm{\gamma})$ if and only if $\bm{\gamma}$ is a stationary point of $g$, because the minorizing function is tangent to $g$ at the current iterate.
It follows that $\lim_{t \to \infty} \bm{\gamma}^{(t)} = \bm{\gamma}^{\star}$.
\end{proof}

How fast does the sequence of iterates converge?
It is known that MM algorithms exhibit geometric convergence in a neighborhood of the maximizer \citep{lange2000optimization}, but a thorough investigation of the convergence properties is left for future work.
%In practice, we notice that adding a little bit of regularization through the Gamma prior greatly improves convergence.

The structure of the updates in~\eqref{cr:eq:mmiter} leads to an extremely simple and efficient implementation, described in Algorithm~\ref{cr:alg:choicerank}: we call it ChoiceRank.
A graphical representation of an iteration from the perspective of a single node is given in Figure~\ref{cr:fig:msgpassing}.
Each iteration consists of two phases of message passing, with $\gamma_i$ flowing towards in-neighbors $\mathcal{N}^-_i$, then $\mu_i$ flowing towards out-neighbors $\mathcal{N}^+_i$ (each message being weighted by the edge strength $w_{ij}$).
The updates to a node's state are a function of the sum of the messages.
As the algorithm does two passes over the edges and two passes over the vertices, an iteration takes $\BigO{M + N}$ time.
The edges can be processed in any order, and the algorithm maintains a state over only $\BigO{N}$ values associated with the vertices.
Furthermore, the algorithm can be conveniently expressed in the well-known vertex-centric programming model \citep{malewicz2010pregel}.
This makes it easy to implement ChoiceRank inside scalable, optimized graph-processing systems such as Apache Spark \citep{gonzalez2014graphx}.

\begin{algorithm}
  \caption{ChoiceRank.}
  \label{cr:alg:choicerank}
  \begin{algorithmic}[1]
    \Require graph $\mathcal{G} = (\mathcal{V}, \mathcal{E})$, counts $\{ (c^-_i, c^+_i) \}$, edge weights $\{ w_{ij} \}$
    \State $\bm{\gamma} \gets [1 \ \cdots \ 1]^\Tr$
    \Repeat
      \State $\bm{z} \gets \bm{0}_N$
      \Comment{Recompute $\bm{\mu}$}
      \OneLineFor{$(i, j) \in \mathcal{E}$} $z_i \gets z_i + w_{ij} \gamma_j$ \label{cr:line:msg1}
      \OneLineFor{$i \in \mathcal{V}$} $\mu_i \gets c^+_i / z_i$
      \State $\bm{z} \gets \bm{0}_N$
      \Comment{Recompute $\bm{\gamma}$}
      \OneLineFor{$(i, j) \in \mathcal{E}$} $z_j \gets z_j + w_{ij} \mu_i$ \label{cr:line:msg2}
      \OneLineFor{$i \in \mathcal{V}$} $\gamma_i \gets (c^-_i + \alpha - 1) / (z_i + \beta)$
    \Until $\bm{\gamma}$ has converged
  \end{algorithmic}
\end{algorithm}

\begin{figure}[t]
  \centering
  \includegraphics[scale=0.8]{cr-message-passing}
  \caption{One iteration of ChoiceRank from the perspective of node $2$.
  Messages flow in both directions along the edges of the graph $\mathcal{G}$, first in the reverse direction (in dotted) then in the forward direction (in solid).}
  \label{cr:fig:msgpassing}
\end{figure}

%%%%%%%%%%%%%%%%%%%%%%%%%%%%%%%%%%%%%%%%%%%%%%%%%%%%%%%%%%%%%%%%%%%%%%%%%
\subsection{EM Viewpoint}

The MM algorithm can also be interpreted from an expectation-maximization (EM) viewpoint, following the ideas of \citet{caron2012efficient}.
We introduce $N$ independent random variables $\mathcal{Z} = \{ z_i : i = 1, \ldots, N \}$, where
\begin{align*}
z_i \sim \text{Gamma} \bigg( c^+_i, \sum_{j \in \mathcal{N}^+_i} w_{ij} \gamma_j \bigg).
\end{align*}
With the addition of these latent random variables, the complete log-likelihood becomes
\begin{align*}
\ell(\bm{\gamma} ; \mathcal{D}, \mathcal{Z})
    &= \ell(\bm{\gamma}; \mathcal{D}) + \sum_{i = 1}^N \log p(z_i \mid \mathcal{D}, \bm{\gamma}) \\
    &= \sum_{i = 1}^N \bigg[ c^-_i \log \gamma_i - c^+_i \log \sum_{k \in \mathcal{N}^+_i} w_{ik} \gamma_k \bigg] \\
    &\qquad +\sum_{i = 1}^N \bigg[  c^+_i \log \sum_{k \in \mathcal{N}^+_i} w_{ik} \gamma_k - z_i \sum_{k \in \mathcal{N}^+_i} w_{ik} \gamma_k \bigg] + \kappa \\
    &= \sum_{i = 1}^N \bigg[ c^-_i \log \gamma_i - z_i \sum_{k \in \mathcal{N}^+_i} w_{ik} \gamma_k \bigg] + \kappa.
\end{align*}
Using a $\text{Gamma}(\alpha, \beta)$ prior for each parameter, the expected value of the log-posterior with respect to the conditional $\mathcal{Z} \mid \mathcal{D}$ under the estimate $\bm{\gamma}^{(t)}$ is
\begin{align*}
Q(\bm{\gamma}, \bm{\gamma}^{(t)})
    &= \mathbf{E}_{\mathcal{Z} \mid \mathcal{D}, \bm{\gamma}^{(t)}} \left[ \ell(\bm{\gamma} ; \mathcal{D}, \mathcal{Z}) \right]
       + \log p(\bm{\gamma}) \\
    &=\sum_{i = 1}^N \bigg[ c^-_i \log \gamma_i - c^+_i \frac{\sum_{k \in \mathcal{N}^+_i} w_{ik} \gamma_k}{\sum_{k \in \mathcal{N}^+_i} w_{ik} \gamma^{(t)}_k} \bigg]
      + \sum_{i = 1}^N \bigg[ (\alpha -1) \log \gamma_i - \beta \gamma_i \bigg] + \kappa.
\end{align*}
The EM algorithm starts with an initial $\bm{\gamma}^{(0)}$ and iteratively refines the estimate by solving the optimization problem $\bm{\gamma}^{(t+1)} = \Argmax_{\bm{\gamma}} Q(\bm{\gamma}, \bm{\gamma}^{(t)})$.
It is not difficult to see that for a given $\bm{\gamma}^{(t)}$, maximizing $Q(\bm{\gamma}, \bm{\gamma}^{(t)})$ is equivalent to maximizing the minorizing function $f^{(t)}(\bm{\gamma})$ defined in~\eqref{cr:eq:minorizing}.
Hence, the MM and the EM viewpoint lead to the exact same sequence of iterates.

The EM formulation leads to a Gibbs sampler in a relatively straightforward way \citep{caron2012efficient}.
We leave a systematic treatment of Bayesian inference in the network choice model for future work.

%%%%%%%%%%%%%%%%%%%%%%%%%%%%%%%%%%%%%%%%%%%%%%%%%%%%%%%%%%%%%%%%%%%%%%%%%
\section{Experimental Evaluation}  %%%%%%%%%%%%%%%%%%%%%%%%%%%%%%%%%%%%%%
\label{cr:sec:experiments}

% Think of the difference between *predictive* and *explanatory*.
In this section, we investigate
\begin{enuminline}
\item the ability of the network choice model to accurately recover transitions in real-world scenarios, and
\item the potential of ChoiceRank to scale to very large networks.
\end{enuminline}

%%%%%%%%%%%%%%%%%%%%%%%%%%%%%%%%%%%%%%%%%%%%%%%%%%%%%%%%%%%%%%%%%%%%%%%%%
\subsection{Accuracy on Real-World Data}
\label{cr:sec:accuracy}

We evaluate the network choice model on three datasets that are representative of two distinct application domains.
%The first dataset contains clickstream data from the English Wikipedia, i.e., traces of users' navigation on a Web site.
%The second dataset consists of the records of all trips made using New York City's bicycle-sharing service during the year 2015, i.e., mobility traces in a large city.
Each dataset can be represented as a set of transition counts $\{ c_{ij} \}$ on a directed graph $\mathcal{G} = (\mathcal{V}, \mathcal{E})$.
We aggregate the transition counts into marginal traffic data $\{ (c^-_i, c^+_i) : i \in \mathcal{V} \}$ and fit a network choice model by using ChoiceRank (for simplicity, we set $w_{ij} \equiv 1$ for all datasets).
We set $\alpha = 2.0$ and $\beta = 1.0$ (these small values simply guarantee the convergence of the algorithm) and declare convergence when $\Norm{\bm{\gamma}^{(t)} - \bm{\gamma}^{(t-1)}} / N < 10^{-8}$.
Given $\bm{\gamma}$, we estimate transition probabilities using $p_{ij} \propto \gamma_j$ as given by \eqref{cr:eq:singlelik}.
To the best of our knowledge, there is no other published method tackling the problem of estimating transition probabilities from marginal traffic data.
Therefore, we compare our method to three baselines based on simple heuristics.
\begin{description}
\item[Traffic] Transitions probabilities are proportional to the traffic of the target node: $q_{ij}^T \propto c_j^{-}$.
\item[PageRank] Transition probabilities are proportional to the PageRank score of the target node: $q_{ij}^P \propto \text{PR}_j$.
\item[Uniform] Any transition is equiprobable: $q_{ij}^U \propto 1$.
\end{description}
The four estimates are compared against ground-truth transition probabilities derived from the edge traffic data: $p_{ij}^\star \propto c_{ij}$.
We emphasize that although per-edge transition counts $\{c_{ij}\}$ are needed to \emph{evaluate} the accuracy of the network choice model (and the baselines), these counts are not necessary for \emph{learning} the model---per-node marginal counts are sufficient.

Given a node $i$, we measure the accuracy of a distribution $\bm{q}_i$ over outgoing transitions using two error metrics, the KL-divergence and the (normalized) rank displacement:
\begin{align*}
D_{\text{KL}}(\bm{p}_i^\star, \bm{q}_i) &= \sum_{j \in \mathcal{N}^+_i} p^\star_{ij} \log \frac{p^\star_{ij}}{q_{ij}}, \\
D_{\text{FR}}(\bm{p}_i^\star, \bm{q}_i) &= \frac{1}{\Abs{\mathcal{N}^+_i}^2} \sum_{j \in \mathcal{N}^+_i} \vert \sigma^\star_i(j) - \hat{\sigma}_i(j) \vert,
\end{align*}
where $\sigma^\star_i$ (respectively $\hat{\sigma}_i$) is the ranking of elements in $\mathcal{N}^+_i$ by decreasing order of $p^\star_{ij}$ (respectively $q_{ij}$).
We report the distribution of errors ``over choices'', i.e., the error at each node $i$ is weighted by the number of outgoing transitions $c^+_i$.


\subsubsection{Clickstream Data}

\paragraph{Wikipedia}
The Wikimedia Foundation has a long history of publicly sharing aggregate, page-level web traffic data\footnote{See: \url{https://stats.wikimedia.org/}.}.
Recently, it also released clickstream data from the English version of Wikipedia \citep{wulczyn2016wikipedia}, providing us with essential ground-truth transition-level data.
We consider a dataset that contains information, extracted from the server logs, about the traffic each page of the English Wikipedia received during the month of March 2016.
Each page's incoming traffic is grouped by HTTP referrer, i.e., by the page visited prior to the request.
We ignore the traffic generated by external Web sites such as search engines and keep only the internal traffic (\num{18}\% of the total traffic in the dataset).
In summary, we obtain counts of transitions on the hyperlink graph of English Wikipedia articles.
The graph contains $N = \num{2316032}$ nodes and $M = \num{13181698}$ edges, and we consider slightly over \num{1.2} billion transitions over the edges.
On this dataset, ChoiceRank converges after \num{795} iterations.

\paragraph{Kosarak}
We also consider a second clickstream dataset from a Hungarian online news portal\footnote{The data is publicly available at \url{http://fimi.ua.ac.be/data/}.}.
The data consists of $\num{7029013}$ transitions on a graph containing $N = 41001$ nodes and $M = \num{974560}$ edges.
ChoiceRank converges after \num{625} iterations.

The four leftmost plots of Figure~\ref{cr:fig:perf-combined} show the error distributions.
ChoiceRank significantly improves on the baselines, both in terms of KL-divergence and rank displacement.
These results give compelling evidence that transitions do not occur proportionally with the target's page traffic: in terms of KL-divergence, ChoiceRank improves on Traffic by a factor $3\times$ and $2\times$, respectively.
PageRank scores, while reflecting some notion of importance of a page, are not designed to estimate transitions, and understandably the corresponding baseline performs poorly.
Uniform (perhaps the simplest of our baselines) is (by design) unable to distinguish among transitions, resulting in a large displacement error.
We believe that its comparatively better performance in terms of KL-divergence (for Wikipedia) is mostly an artifact of the metric, which encourages ``prudent'' estimates.
Finally, in Figure~\ref{cr:fig:perf-wikipedia} we observe that ChoiceRank seems to perform comparatively better as the number of possible transition increases.

\begin{figure}
  \centering
  \includegraphics{cr-perf-combined}
  \caption{
Error distributions of the network choice model and three baselines for the Wikipedia, Kosarak and Citi Bike datasets.
The boxes show the interquartile range, the whiskers show the $5^{\text{th}}$ and $95^{\text{th}}$ percentiles, the red horizontal bars show the median and the red squares show the mean.
}
  \label{cr:fig:perf-combined}
\end{figure}

\begin{figure}
  \centering
  \includegraphics{cr-perf-wikipedia}
  \caption{
Average KL-divergence as a function of the number of possible transitions for the Wikipedia dataset.
ChoiceRank performs comparatively better in the case where a node's out-degree is large.
}
  \label{cr:fig:perf-wikipedia}
\end{figure}


\subsubsection{NYC Bicycle-Sharing Data}

Next, we consider trip data from Citi Bike, New York City's bicycle-sharing system\footnote{The data is available at \url{https://www.citibikenyc.com/system-data}.}.
%Markov models have been used with success in the context of mobility prediction \citep{ashbrook2003using, kafsi2015traveling}. TODO
For each ride on the system made during the year 2015, we extract the pick-up and drop-off stations and the duration of the ride.
Because we want to focus on direct trips, we exclude rides that last more than one hour.
We also exclude source-destinations pairs which have less than 1 ride per day on average (a majority of source-destination pairs appears at least once in the dataset).
The resulting data consists of \num{3.4} million rides on a graph containing $N = \num{497}$ nodes and $M = \num{5209}$ edges.
ChoiceRank converges after $\num{7508}$ iterations.
We compute the error distribution in the same way as for the clickstream datasets.

The two rightmost plots of Figure~\ref{cr:fig:perf-combined} display the results.
The observations made on the clickstream datasets carry over to this mobility dataset, albeit to a lesser degree.
A significant difference between clicking a link and taking a bicycle trip is that in the latter case, there is a non-uniform ``cost'' of a transition due to the distance between source and target.
In future work, one might consider experimenting with edge weights $\{ w_{ij} \}$ that capture this.


%%%%%%%%%%%%%%%%%%%%%%%%%%%%%%%%%%%%%%%%%%%%%%%%%%%%%%%%%%%%%%%%%%%%%%%%%
\subsection{Scaling to Large Networks}

To demonstrate ChoiceRank's scalability, we develop a simple implementation in the Rust programming language, based on the ideas of COST \citep{mcsherry2015scalability}.
Our code is publicly available online\footnote{See: \url{http://lucas.maystre.ch/choicerank}.}.
The implementation repeatedly streams edges from disk and keeps four floating-point values per node in memory:
the counts $c^-_i$ and $c^+_i$, the sum of messages $z_i$, and either $\mu_i$ or $\gamma_i$ (depending on the stage in the iteration).
As edges can be processed in any order, it can be beneficial to reorder the edges in a way that accelerates the computation.
For this reason, our implementation preprocesses the list of edges and reorders them in Hilbert curve order\footnote{A Hilbert space-filling curve visits all the entries of the adjacency matrix of the graph, in a way that preserves locality of both source and destination of the edges.}.
This results in better cache locality and yields a significant speedup.

We test our implementation on a hyperlink graph extracted from the 2012 Common Crawl web corpus\footnote{
The data is available at \url{http://webdatacommons.org/hyperlinkgraph/}.} that contains over \num{3.5} billion nodes and \num{128} billion edges \citep{meusel2014graph}.
The edge list alone requires about $1$ TB of uncompressed storage.
There is no publicly available information on the traffic at each page, therefore we generate a value $c_i$ for every node $i$ randomly and uniformly between \num{100} and \num{500}, and set both $c^-_i$ and $c^+_i$ to $c_i$.
As such, this experiment does not attempt to measure the validity of the model (unlike the experiments of Section~\ref{cr:sec:accuracy}).
Instead, it focuses on testing the algorithm's potential to scale to to very large networks.

\paragraph{Results}
We run \num{20} iterations of ChoiceRank on a dual Intel Xeon E5-2680 v3 machine, with \num{256} GB of RAM and \num{6} HDDs configured in RAID 0.
We arbitrarily set $\alpha = 2.0$ and $\beta = 1.0$ (but this choice has no impact on the results).
Only about \num{65} GB of memory is used, all to store the nodes' state ($4 \times 4$ bytes per node).
The algorithm takes a little less than \num{39} minutes per iteration on average.
Collectively, these results validate the feasibility of model inference for very large datasets.

It is worth noting that despite tackling different problems, the ChoiceRank algorithm exhibits interesting similarities with a message-passing implementation of PageRank commonly used in scalable graph-parallel systems such as Pregel \citep{malewicz2010pregel} and Spark \citep{gonzalez2014graphx}.
For comparison, using the COST code \citep{mcsherry2015scalability} we run \num{20} iterations of PageRank on the same hardware and data.
PageRank uses slightly less memory (about \num{50} GB, or one less floating-point number per node) and takes about half of the time per iteration (a little over \num{20} minutes).
This is consistent with the fact that ChoiceRank requires two passes over the edges per iteration, whereas PageRank requires one.
The similarities between the two algorithms lead us to believe that in general, ChoiceRank can benefit from any new system optimizations developed for PageRank.

%%%%%%%%%%%%%%%%%%%%%%%%%%%%%%%%%%%%%%%%%%%%%%%%%%%%%%%%%%%%%%%%%%%%%%%%%
\section{Summary}  %%%%%%%%%%%%%%%%%%%%%%%%%%%%%%%%%%%%%%%%%%%%%%%%%%%%%%
\label{cr:sec:summary}

In this chapter, we present a method that tackles the problem of finding the transition probabilities along the edges of a network, given only the network's structure and aggregate node-level traffic data.
This method generalizes and extends ideas recently presented by \citet{kumar2015inverting}.
We demonstrate that in spite of the strong model assumptions needed to learn $\BigO{N^2}$ probabilities from $\BigO{N}$ observations, the method still manages to recover the transition probabilities to a good level of accuracy on two clickstream datasets, and shows promise for applications beyond clickstream data.
To sum up, we believe that our method will be useful to pracitioners interested in understanding patterns of navigation in networks from aggregate traffic data, commonly available, e.g., in public datasets.

