\documentclass[12pt,a4paper,oneside]{article}
\usepackage[T1]{fontenc} % Use 8-bit output encoding.
\usepackage[utf8]{inputenc} % Can use UTF-8 in the source files.
\usepackage{natbib}  % Has to be loaded before babel.
\usepackage[english]{babel} % English language/hyphenation.
\usepackage[babel]{microtype} % Improves appearance of text.
\usepackage{fullpage}
\usepackage{amsmath,amssymb,amsthm,bm,mathtools}
\usepackage{algorithm}
\usepackage{algpseudocode}

\theoremstyle{plain}
\newtheorem{proposition}{Proposition}
\newtheorem{corollary}{Corollary}

\begin{document}

\section{A Convergent Power-Iteration Scheme}

First we show that the dynamic range of the ML estimate of BT model parameters admits a simple upper bound.

\begin{proposition}
Let $\mathcal{D} = \{ a_{ij} \mid i, j = 1, \ldots N \}$ be a dataset of outcomes of pairwise comparison on $N$ items.
Suppose that the comparison graph is strongly connected, and define $S = \sum_{i, j} a_{ij}$.
Let $\bm{\theta} = \begin{bmatrix}\theta_1 & \cdots & \theta_N \end{bmatrix}$ be the maximum-likelihood estimate of Bradley--Terry model parameters.
Then,
\begin{align*}
\max_{i, j} \ \lvert \theta_i - \theta_j \rvert \le N \log S.
\end{align*}
\end{proposition}

\begin{proof}
Without loss of generality, suppose that $\theta_1 \le \theta_2 \le \ldots \le \theta_N$.
We will show that $\theta_i \ge \theta_{i+1} - \log S$ for all $i$, and the proposition follows.
By contradiction, suppose that $\theta_k = \theta_{k+1} - \log S - \varepsilon$ for some $k \in [N]$ and some $\varepsilon > 0$.
Let
\begin{align*}
\tilde{\bm{\theta}} =
\begin{bmatrix}
\theta_1 + \varepsilon & \cdots & \theta_k + \varepsilon & \theta_{k+1} & \cdots & \theta_N
\end{bmatrix}.
\end{align*}
Using basic algebraic manipulations, we find that the difference between the log-likelihoods can be written as
\begin{align*}
\ell(\tilde{\bm{\theta}}) - \ell(\bm{\theta}) =
    \sum_{i=1}^k \sum_{j = k+1}^N a_{ji} \varepsilon
  - \sum_{i=1}^k \sum_{j = k+1}^N (a_{ij} + a_{ji})
        \log \left[ \frac{\exp(\theta_i + \varepsilon) + \exp \theta_j}{\exp \theta_i + \exp \theta_j} \right].
\end{align*}
Because the comparison graph is strongly connected, $\sum_{i=1}^k \sum_{j = k+1}^N a_{ji} \ge 1$.
Furthermore, $\sum_{i=1}^k \sum_{j = k+1}^N (a_{ij} + a_{ji}) < S$.
Finally, for all $j > k$,
\begin{align}
\log \left[ \frac{\exp(\theta_i + \varepsilon) + \exp \theta_j}{\exp \theta_i + \exp \theta_j} \right]
    &\le \frac{\exp(\theta_i + \varepsilon) - \exp \theta_i}{\exp \theta_i + \exp \theta_j}
       = \frac{1 - \exp(-\varepsilon)}{\exp(-\varepsilon) + \exp(\theta_j - \theta_i - \varepsilon)} \label{eq:line1}\\
    &\le \frac{1 - \exp(-\varepsilon)}{\exp(-\varepsilon) + S}
       < \frac{\varepsilon}{S}, \label{eq:line2}
\end{align}
where in~\eqref{eq:line1} we used the inequality $\log(x) < x - 1$, and in~\eqref{eq:line2} we used the fact that $\theta_j \ge \theta_i + \log S + \varepsilon$ and the inequality $1 - e^x \le x$.
Putting everything together, we find that
\begin{align*}
\ell(\tilde{\bm{\theta}}) - \ell(\bm{\theta}) > \varepsilon - S \frac{\varepsilon}{S} = 0,
\end{align*}
and conclude that $\bm{\theta}$ cannot be the maximum-likelihood estimate of the parameters.
\end{proof}

\begin{corollary}
\label{thm:pidynrange}
Let $\bm{\pi} = \begin{bmatrix} \pi_1 & \cdots & \pi_N \end{bmatrix}$ be the reparametrized maximum-likelihood estimate such that $\pi_i \propto \exp(\theta_i)$.
Furthermore, assume that $\bm{\pi}$ is normalized such that $\sum_i \pi_i = 1$.
Then,
\begin{align*}
\max_{i, j} \pi_i / \pi_j \le S^N,
\qquad \min_i \pi_i \ge S^{-N} / N.
\end{align*}
\end{corollary}
\begin{proof}
The first result comes from the fact that $\pi_i / \pi_j = \exp(\theta_i - \theta_j)$, and the second one from the fact that $\max_i \pi_i \ge 1/N$.
\end{proof}

We are now ready to present a power-iteration based algorithm that is provably convergent.
Let $\mathcal{S} = \{ \bm{\pi} \in \mathbf{R}^N : 0 < \pi_i < 1, \sum_i \pi_i = 1 \}$ be the open probability simplex and define $\widetilde{\mathcal{S}} = \{ \bm{\pi} \in \mathcal{S} : \pi_i \ge S^{-N}/N \}$.
By Corollary~\ref{thm:pidynrange}, this compact (closed) subset is guaranteed to contain the maximum-likelihood estimate.
Let
\begin{align}
\label{eq:transmat}
P(\varepsilon, \bm{\pi})_{ij} =
\begin{dcases}
\varepsilon \frac{a_{ij}}{\pi_i + \pi_j}                    & \text{if } i \ne j, \\
1 - \varepsilon \sum_{k \ne i} \frac{a_{ik}}{\pi_i + \pi_k} & \text{if } i = j,
\end{dcases}
\end{align}
be the (parametric) transition matrix of a Markov chain, for sufficiently small values of $\varepsilon$.
Lastly, define $\text{Proj}: \mathcal{S} \to \widetilde{\mathcal{S}}$ as
\begin{align*}
\text{Proj}[\bm{\pi}]_i = \max \{ S^{-N}/N, \pi_i \}.
\end{align*}

\begin{algorithm}[t]
  \caption{A provably-convergent algorithm based on the power method.}
  \label{alg:lsrpower}
  \begin{algorithmic}[1]
    \Require learning rate $\varepsilon$.
    \State $\bm{\pi} = \begin{bmatrix}1/N & \cdots & 1/N\end{bmatrix}$
    \Repeat
      \For{$i = 1, \ldots, N$} \label{line:startpow}
        \State $Q \gets P(\varepsilon, \bm{\pi})$ as per~\eqref{eq:transmat}
        \State $\bm{\pi} \gets \bm{\pi}^\top Q$
      \EndFor \label{line:stoppow}
      \State $\bm{\pi} \gets \text{Proj}[\bm{\pi}]$
    \Until convergence
  \end{algorithmic}
\end{algorithm}

\begin{proposition}
Algorithm~\ref{alg:lsrpower} converges to the maximum-likelihood estimate for any fixed $\varepsilon \le 1 / (2N^2S^{N+1})$.
\end{proposition}

\begin{proof}[Proof (sketch).]
Let $T : \widetilde{\mathcal{S}} \to \mathcal{S}$ be the mapping corresponding to the inner loop (the $N$ power iterations, lines \ref{line:startpow}--\ref{line:stoppow}).
By choosing $\varepsilon \le 1 / (2N^2S^{N+1})$ we ensure that the matrix $P(\bm{\pi}, \varepsilon)$ remains stochastic (i.e., with all entries nonnegative) throughout the $N$ iterations.
Indeed, letting $\bm{\pi}^{(t)}$ represent the $t$-th power iterate, we have that for all $i$,
\begin{align*}
&\forall j \ \pi_j^{(t)} > S^{-N}/(2N) \\
&\quad \implies
\pi_i^{(t+1)} \ge \left( 1 - \varepsilon \sum_{k \ne i} \frac{a_{ik}}{\pi_i^{(t)} + \pi_k^{(t)}} \right) \pi_i^{(t)} > \left( 1 - \varepsilon \frac{S}{S^{-N}/N} \right) \pi_i^{(t)} \ge \left( 1 - \frac{1}{2N} \right) \pi_i^{(t)},
\end{align*}
Hence, starting with $\bm{\pi}^{(0)} \in \widetilde{\mathcal{S}}$, it is easy to verify that for $t = 1, \ldots, N$,
\begin{align*}
\pi_i^{(t)} > \left( 1 - \frac{1}{2N} \right)^t \pi_0^{(t)} \ge \left( 1 - \frac{t}{2N} \right) \pi_0^{(t)} \ge S^{-N}/(2N),
\end{align*}
and the diagonal entries of $P(\bm{\pi}, \varepsilon)$ remain always nonnegative.
Furthermore, doing $N$ iterations ensures that for any $\bm{x}, \bm{y} \in \widetilde{\mathcal{S}}$,
\begin{align}
\lVert T(\bm{x}) - T(\bm{y}) \rVert_1 < \lVert \bm{x} - \bm{y} \rVert_1.
\end{align}
(Note that a single power-iteration is not sufficient to obtain a contractive property with respect to the $\ell_1$-norm.)
In order to show this, it is necessary to compute the Jacobian of $T(\bm{x})$ using the chain rule;
see \citet[Lemma~$2.17$]{olver2015nonlinear}, \citet[Theorem~$2$]{petersdorff2014fixed} and \citet{tresch2007convergence}.
Now, consider the mapping $F : \widetilde{\mathcal{S}} \to \widetilde{\mathcal{S}}$ defined by the outer loop,
\begin{align*}
F(\bm{\pi}) = \text{Proj}[T(\bm{\pi})].
\end{align*}
The projection operator is non-expansive with respect to the $\ell_1$ norm, and therefore for any $\bm{x}, \bm{y} \in \widetilde{\mathcal{S}}$,
\begin{align*}
\lVert F(\bm{x}) - F(\bm{y}) \rVert_1 \le \lVert T(\bm{x}) - T(\bm{y}) \rVert_1 < \lVert \bm{x} - \bm{y} \rVert_1.
\end{align*}
Hence, $F$ is a contractive mapping on a compact space, and a variant of Banach's fixed point principle \citep[Theorem~2.6]{kirk2001contraction} implies that Algorithm~\ref{alg:lsrpower} converges to the unique fixed point $\hat{\bm{\pi}}$.
\end{proof}

Note that it might not be necessary to take multiple power-iterations before the projection step.
See Corollary~$2.13$ in \citet{olver2015nonlinear} for inspiration.

\bibliographystyle{abbrvnat}
\bibliography{notes.bib}

\end{document}
